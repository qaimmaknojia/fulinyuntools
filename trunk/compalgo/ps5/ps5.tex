\documentclass[12pt,letterpaper]{article}
\usepackage[latin1]{inputenc}
\usepackage{amsmath}
\usepackage{amsfonts}
\usepackage{amssymb}
\usepackage{graphicx}

\newtheorem{clm}{Claim}

\author{Linyun Fu}
\title{CSCI 4020 Computer Algorithms Spring 2011\\
Problem Set 5}
\begin{document}
\maketitle
\section*{Chapter 7, Problem 7}
Consider a set of mobile computing clients in a certain town who each
need to be connected to one of several possible \emph{base stations}. We'll
suppose there are $n$ clients, with the position of each client specified
by its $(x, y)$ coordinates in the plane. There are also $k$ base stations; the
position of each of these is specified by $(x, y)$ coordinates as well.

For each client, we wish to connect it to exactly one of the base
stations. Our choice of connections is constrained in the following ways.
There is a range parameter $r$ -- a client can only be connected to a base
station that is within distance $r$. There is also a \emph{load parameter} $L$ -- no
more than $L$ clients can be connected to any single base station.

Your goal is to design a polynomial-time algorithm for the following
problem. Given the positions of a set of clients and a set of base stations,
as well as the range and load parameters, decide whether every client can
be connected simultaneously to a base station, subject to the range and
load conditions in the previous paragraph.

\section*{Answer}
We denote the $n$ client nodes with $c_1, c_2, \dots, c_n$ and the $k$ base stations with $b_1, b_2, \dots, b_k$. We convert the problem into a maximum-flow problem in the following way.
\begin{itemize}
\item Add a source node $s$ and add an edge from it to each client node with capacity 1;
\item For each pair $(c_i, b_j)$, if the distance between $c_i$ and $b_j$ is within $r$, then add an edge from $c_i$ to $b_j$ with capacity 1;
\item Add a sink node $t$ and add an edge from each base station node to it with capacity $L$.
\end{itemize}

Then we find the maximum flow in this flow network. If the the size of the maximum flow is $n$, then it is possible to connect every client to a base station at the same time; otherwise it is impossible to achieve the goal. Figure~1 illustrates the above conversion scheme.

\begin{figure}
\begin{center}
\includegraphics[width=0.7\textwidth]{7.7.eps}
\caption{Reducing the problem to a maximum-flow problem}
\end{center}
\end{figure}

Next we prove that each solution to the problem corresponds to a maximum flow of size $n$ in the flow network and vice versa.

\begin{itemize}
\item Given a solution to the original problem, we know that each $c_i$ is connected to exactly one $b_j$, and each $b_j$ is connected from at most $L$ $c_i$'s. In the flow network, we let the edge from $c_i$ to $b_j$ carry flow of size 1 if $b_j$ is $c_i$'s base station, we also let the edge from $s$ to each $c_i$ carry flow of size 1 to meet conservation conditions. For the same reason, we let each edge from $b_j$ to $t$ carry flow of size $f^{\textrm{in}}b_j\le L$. Therefore we have a maximum flow of size $n$ since every edge going out from $s$ is saturated.
\item Given a maximum flow of size $n$ in the flow network, we know that each edge out from $s$ must be saturated, and each edge in to $t$ must carry a flow no more than $L$, so $f^{\textrm{in}}c_i=1$ for each $c_i$ and $f^{\textrm{out}}b_j\le L$ for each $b_j$, thus $f^{\textrm{out}}c_i=f^{\textrm{in}}c_i=1$ for each $c_i$, and $f^{\textrm{in}}b_j = f^{\textrm{out}}b_j\le L$ for each $b_j$, which means all the edges between $\{c_1, c_2, \dots, c_n\}$ and $\{b_1, b_2, \dots, b_k\}$ carrying flow of size 1 make a valid solution to the original problem.
\end{itemize}

The conversion is composed of adding nodes $s$ and $t$, connecting $s$ with $n$ nodes and $t$ with $k$ nodes, and assigning capacity values to all the edges in the flow network, which can be completed in polynomial time; the maximum-flow problem can also be solved in polynomial time. Therefore, the time complexity of the algorithm is polynomial.

\section*{Chapter 7, Problem 34}
\emph{Ad hoc networks}, made up of low-powered wireless devices, have been
proposed for situations like natural disasters in which the coordinators
of a rescue effort might want to monitor conditions in a hard-to-reach
area. The idea is that a large collection of these wireless devices could be
dropped into such an area from an airplane and then configured into a
functioning network.

Note that we're talking about (a) relatively inexpensive devices that
are (b) being dropped from an airplane into (c) dangerous territory; and
for the combination of reasons (a), (b), and (c), it becomes necessary to
include provisions for dealing with the failure of a reasonable number of
the nodes.

We'd like it to be the case that if one of the devices $v$ detects that it is in
danger of failing, it should transmit a representation of its current state to
some other device in the network. Each device has a limited transmitting
range -- say it can communicate with other devices that lie within $d$ meters
of it. Moreover, since we don't want it to try transmitting its state to a
device that has already failed, we should include some redundancy: A
device $v$ should have a set of $k$ other devices that it can potentially contact,
each within $d$ meters of it. We'll call this a \emph{back-up set} for device $v$.

\begin{itemize}
\item[(a)] Suppose you're given a set of $n$ wireless devices, with positions
represented by an $(x, y)$ coordinate pair for each. Design an algorithm
that determines whether it is possible to choose a back-up set for
each device (i.e., $k$ other devices, each within $d$ meters), with the
further property that, for some parameter $b$, no device appears in
the back-up set of more than $b$ other devices. The algorithm should
output the back-up sets themselves, provided they can be found.

\item[(b)] The idea that, for each pair of devices $v$ and $w$, there's a strict
dichotomy between being ``in range'' or ``out of range'' is a simplified
abstraction. More accurately, there's a power decay function $f(\cdot)$ that
specifies, for a pair of devices at distance $\delta$, the signal strength $f(\delta)$
that they'll be able to achieve on their wireless connection. (We'll
assume that $f(\delta)$ decreases with increasing $\delta$.)

We might want to build this into our notion of back-up sets as
follows: among the $k$ devices in the back-up set of $v$, there should
be at least one that can be reached with very high signal strength,
at least one other that can be reached with moderately high signal
strength, and so forth. More concretely, we have values $p_1 \ge p_2 \ge\cdots \ge
p_k$, so that if the back-up set for $v$ consists of devices at distances
$d_1 \le d_2 \le \cdots\le d_k$, then we should have $f(d_j) \ge p_j$ for each $j$.

Give an algorithm that determines whether it is possible to
choose a back-up set for each device subject to this more detailed
condition, still requiring that no device should appear in the back-up
set of more than $b$ other devices. Again, the algorithm should output
the back-up sets themselves, provided they can be found.
\end{itemize}

\section*{Answer}

\subsection*{Problem (a)}
We convert the original problem into a maximum-flow problem in the following way.
\begin{itemize}
\item For each wireless device, create two nodes, namely $v_i$ and $v_i'$ ($i=1, 2, \dots, n$);
\item Create the source node $s$ and the sink node $t$;
\item For each node $v_i$, create an edge from $s$ to $v_i$ with capacity $k$;
\item For every pair $(v_i, v_j')$ such that $i \ne j$, if the distance between device $i$ and $j$ is within $d$, then create an edge from $v_i$ to $v_j'$ with capacity 1;
\item For each node $v_j'$, create an edge from $v_j'$ to $t$ with capacity $b$.
\end{itemize}

Then we find the maximum flow in the this flow network. If the size of the maximum flow is not $nk$, then we know that the required back-up sets do not exist; otherwise the edges between $\{v_1, v_2, \dots, v_n\}$ and $\{v_1', v_2', \dots, v_n'\}$ that carry flow of size 1 make a valid solution to the original problem, i.e., if edge $(v_i, v_j')$ carries flow of size 1, then device $j$ is in device $i$'s back-up set. Figure~2 illustrates the above conversion.

\begin{figure}
\begin{center}
\includegraphics[width=0.7\textwidth]{7.34a.eps}
\caption{Reducing problem (a) to maximum-flow problem}
\end{center}
\end{figure}

Next we prove that each solution to the problem corresponds to a maximum flow of size $nk$ in the flow network and vice versa.
\begin{itemize}
\item Given a solution to the original problem, we know that each device finds $k$ other devices within a distance of $d$, and no device is in the back-up sets of more than $b$ other devices. In the flow network, we let the edge from $v_i$ to $v_j'$ carry flow of size 1 if device $i$ finds $j$ as one of its back-up devices. We then let the edge from $s$ to each $v_i$ carry flow of size $k$ to meet conservation conditions, because $f^{\textrm{out}}v_i = k$ for each $i$. For the same reason, we let the edge from $v_j'$ to $t$ carry flow of size $f^{\textrm{in}}v_j'$ for each $j$. Since no device is in the back-up sets of more than $b$ other devices, no edge from $v_j'$ to $t$ carry flow of size more than $b$. Therefore, it is a maximum flow of size $nk$ in the flow network since every edge going out from $s$ is saturated.
\item Given a maximum flow of size $nk$ in the flow network, we let device $j$ in $i$'s back-up set if the edge from $v_i$ to $v_j'$ carries flow of size 1. We know that every edge out from $s$ must be saturated, so $f^{\textrm{in}}v_i=k$ for each $i$, therefore $f^{\textrm{out}}v_i=f^{\textrm{in}}v_i=k$ for each $i$,  which means each device finds its $k$ back-up devices,  and the fact that $f^{\textrm{out}}v_j'\le b$ for each $j$ yields that $f^{\textrm{in}}v_j'=f^{\textrm{out}}v_j'\le b$ for each $j$, which means no device appears in the back-up sets of more than $b$ other devices. 
\end{itemize}

The conversion consists of creating $2n+2$ nodes and $O(n^2)$ edges with designated capacities, which can be computed in polynomial time. Since the maximum-flow problem is solvable in polynomial time, the overall time complexity is polynomial.

\subsection*{Problem (b)}
For the sake of simplicity, we made the following definition about \emph{level-$l$ back-up devices}. i.e., if device $j$ is a back-up device for device $i$ and the distance between $i$ and $j$, which is $\delta$, satisfies $p_l\le f(\delta)< p_{l-1}$ ($p_0=+\infty$), then we say $j$ is a \emph{level-$l$ back-up} of $i$. The highest level is level 1 since it has the most strict constraint on distance. The original problem is equivalent to the following problem.
\begin{quote}
Determine whether each device can find at least one level-1 back-up, at least two back-ups of level 2 or above, $\dots$, and at least $k-1$ back-ups of level-$(k-1)$ or above, and totally $k$ back-ups of level-$k$ or above; no device is allowed in the back-up sets of more than $b$ other devices.
\end{quote}

We convert the original problem into a maximum-flow problem in the following way.
\begin{itemize}
\item For each wireless device $i$, create $3k+1$ nodes, namely $u_{il}$, $v_{il}$ and $v_{il}'$  for $l = 1, 2, \dots, k$, and $v_i'$. We use $S_l$ to denote the subgraph containing all the nodes $v_{il}$ and $v_{il}'$ ($i = 1, 2, \dots, n$) hereafter;
\item Create the source node $s$ and the sink node $t$;
\item For each node $u_{i1}$, create an edge from $s$ to $u_{i1}$ with capacity $k$;
\item For each node $u_{il}$, create an edge from $u_{il}$ to $v_{il}$ with capacity $k+1-l$, i.e., $u_{i1}$ to $v_{i1}$ with capacity $k$, $u_{i2}$ to $v_{i2}$ with capacity $k-1$, $\dots$, $u_{ik}$ to $v_{ik}$ with capacity $1$;
\item Create an edge from $u_{il}$ to $u_{i(l+1)}$ with capacity $k-l$ for each $l$ from 1 to $k-1$, i.e., $u_{i1}$ to $u_{i2}$ with capacity $k-1$, $u_{i2}$ to $u_{i3}$ with capacity $k-2$, $\dots$, $u_{i(k-1)}$ to $u_{ik}$ with capacity $1$;
\item For each pair $(v_{i1}, v_{j1}')$ such that $i\ne j$ in $S_1$, if the distance between $v_{i1}$ and $v_{j1}'$, $\delta$, satisfies $f(\delta)\ge p_1$, then create an edge from $v_{i1}$ to $v_{j1}'$ with capacity 1;
\item For each subgraph $S_l$ in $\{S_2, S_3, \dots, S_k\}$, create edges between $v_{il}$'s and $v_{jl}'$'s satisfying $p_l\le f(\delta)<p_{l-1}$ with capacity 1, where $\delta$ is the distance between some pair $(v_{il}, v_{jl}')$;
\item For each node $v_{jl}'$, create an edge from $v_{jl}'$ to $v_j'$ with capacity $n$ (or infinity);
\item For each node $v_j'$, create an edge from $v_j'$ to $t$ with capacity $b$.
\end{itemize}

Then we find the maximum flow in the this flow network. If the size of the maximum flow is not $nk$, then we know that the required back-up sets do not exist; otherwise the edges in $S_1, S_2, \dots, S_k$ that carry flow of size 1 make a valid solution to the original problem, i.e., edges in $S_l$ indicate each device's level-$l$ back-up device(s). Figure~3 illustrates the above conversion.

\begin{figure}
\begin{center}
\includegraphics[width=\textwidth]{7.34b.eps}
\caption{Reducing problem (b) to maximum-flow problem}
\end{center}
\end{figure}

Next we prove that each solution to the problem corresponds to a maximum flow of size $nk$ in the flow network and vice versa.

\begin{itemize}
\item Given a solution to the original problem, we know that each device finds 
\begin{itemize}
\item at least one level-1 back-up, 
\item at least two back-ups of level 2 or above, 

$\dots$, 

\item at least $k-1$ back-ups of level $k-1$ or above, and 
\item totally $k$ back-ups of level $k$ or above, 
\end{itemize}
and no device is in the back-up sets of more than $b$ other devices. We let the edge from $v_{il}$ to $v_{jl}'$ (in subgraph $S_l$) carry flow of size 1 if device $j$ is one of device $i$'s level-$l$ back-ups. So for each $i=1,2,\dots,n$, we have
\begin{align}
\nonumber f^{\textrm{out}}v_{i1} & \ge 1\\\nonumber
f^{\textrm{out}}v_{i1}+f^{\textrm{out}}v_{i2} & \ge 2\\\nonumber
\dots & \dots \dots \\\nonumber
f^{\textrm{out}}v_{i1}+f^{\textrm{out}}v_{i2}+\dots+f^{\textrm{out}}v_{i(k-1)} & \ge k-1\\\nonumber
f^{\textrm{out}}v_{i1}+f^{\textrm{out}}v_{i2}+\dots+f^{\textrm{out}}v_{ik} & = k
\end{align}
therefore 
\begin{align}
\nonumber f^{\textrm{out}}v_{ik} & \le 1\\\nonumber
f^{\textrm{out}}v_{i(k-1)} + f^{\textrm{out}}v_{i(k)}& \le 2\\\nonumber
\dots & \dots \dots\\\nonumber
f^{\textrm{out}}v_{i2} + f^{\textrm{out}}v_{i3} + \dots + f^{\textrm{out}}v_{i(k)} & \le k-1\\\nonumber
f^{\textrm{out}}v_{i1}+f^{\textrm{out}}v_{i2}+\dots+f^{\textrm{out}}v_{ik} & = k
\end{align}
Therefore we can let each edge from $u_{il}$ to $v_{il}$ carry flow of size $f^{\textrm{out}}v_{il}$ and each edge from $u_{il}$ to $u_{i(l+1)}$ carry flow of size $f^{\textrm{out}}u_{i(l+1)}$ to meet conservation conditions while still meeting capacity conditions, and we have $f^{\textrm{out}}u_{ik} = k$ for all $i$. Then we let the edge from $s$ to each $u_{ik}$ carry flow of size $k$ to meet conservation conditions. We also let the edge from $v_{jl}'$ to $v_j'$ carry flow of size $f^{\textrm{in}}v_{jl}'$ for the same reason. Since no device is in the back-up sets of more than $b$ other devices, we have $f^{\textrm{in}}v_{j}' \le b$ for all $j$, so we can let each edge from  $v_j'$ to $t$ carry flow of size $f^{\textrm{in}}v_{j}'$ to meet conservation conditions while still meeting capacity conditions. Now we have a flow saturating all the edges out from $s$, which is a maximum flow of size $nk$.
\item Given a maximum flow of size $nk$ in the flow network, we let device $j$ in device $i$'s back-up set if the edge from $v_{il}$ to $v_{jl}'$ carries flow of size 1 for some $l$. Now we prove this is a valid solution. We know that every edge out from $s$ must be saturated, so $f^{\textrm{in}}u_{il}=k$ for each $i$, therefore $f^{\textrm{out}}u_{il}=k$ for each $i$, so 
\begin{align*}
f^{\textrm{in}}v_{i1}+f^{\textrm{in}}v_{i2}+\dots+f^{\textrm{in}}v_{ik} = k
\end{align*} 
for each $i$, so 
\begin{align*}
f^{\textrm{out}}v_{i1}+f^{\textrm{out}}v_{i2}+\dots+f^{\textrm{out}}v_{ik} = k
\end{align*} 
which means every device finds its $k$ back-up devices. Since 
\begin{align*}
f^{\textrm{out}}v_{i(k-l+1)} + f^{\textrm{out}}v_{i(k-l+2)} + \dots + f^{\textrm{out}}v_{i(k-1)} + f^{\textrm{out}}v_{ik} \le l
\end{align*}
for each $l$, there are at least $k-l$ back-up devices of level $k-l$ or above. On the other hand, since $f^{\textrm{in}}v_j'=f^{\textrm{out}}v_j'\le b$ for each $j$, and $f^{\textrm{in}}v_j'$ is the total number of times that device $j$ acts as some other device's back-up, this means no device appears in the back-up sets of more than $b$ other devices. 
\end{itemize}

The conversion consists of creating $3kn+n+2$ nodes and $O(kn^2)$ edges with designated capacities, which can be computed in polynomial time. Since the maximum-flow problem is solvable in polynomial time, the overall time complexity is polynomial.

\end{document}
