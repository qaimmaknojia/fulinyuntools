\documentclass[12pt,letterpaper]{article}
\usepackage[latin1]{inputenc}
\usepackage{amsmath}
\usepackage{amsfonts}
\usepackage{amssymb}
\usepackage{graphicx}

\newtheorem{clm}{Claim}

\author{Linyun Fu}
\title{CSCI 4020 Computer Algorithms Spring 2011\\
Problem Set 3}
\begin{document}
\maketitle
\section*{Chapter 6, Problem 11}
Suppose you're consulting for a company that manufactures PC equipment
and ships it to distributors all over the country. For each of the
next $n$ weeks, they have a projected supply $s_i$ of equipment (measured in
pounds), which has to be shipped by an air freight carrier.

Each week's supply can be carried by one of two air freight companies,
A or B.
\begin{itemize}
\item Company A charges a fixed rate $r$ per pound (so it costs $r\cdot s_i$ to ship
a week's supply $s_i$).
\item Company B makes contracts for a fixed amount $c$ per week, independent
of the weight. However, contracts with company B must be made
In blocks of four consecutive weeks at a time.
\end{itemize}

A \emph{schedule}, for the PC company, is a choice of air freight Company
(A or B) for each of the $n$ weeks, with the restriction that company B,
whenever it is chosen, must be chosen for blocks of four contiguous
weeks at a time. The \emph{cost} of the schedule is the total amount paid to
company A and B, according to the description above.

Give a polynomial-time algorithm that takes a sequence of supply values
$s_1, s_2, ..., s_n$ and returns a \emph{schedule} of minimum cost.\\
\textbf{Example.} Suppose $r = 1, c = 10$, and the sequence of values is
\begin{align*}
11, 9, 9, 12, 12, 12, 12, 9, 9, 11.
\end{align*}
Then the optimal schedule would be to choose company A for the first
three weeks, then company B for a block of four consecutive weeks, and
then company A for the final three weeks.

\section*{Answer}
We define $MINCOST(i)$ to be the lowest cost that could be achieved to ship the equipment for the first $i$ weeks, and $OPT(i,j)$ to be the company chosen for the $j$-th week to achieve $MINCOST(i)$. Since the best schedule for the first $i$ weeks is either achieved through choosing company A for the $i$-th week, or choosing company B for the $i-3, i-2, i-1$ and $i$-th week, we have
\begin{align}
MINCOST(i) = \min\{MINCOST(i-1)+r\cdot s_i, & & \\\nonumber
 MINCOST(i-4)+4c\} & & \textrm{  for } i \ge 4
\end{align}
but for $i<4$, we have to choose company A for each week:
\begin{align}
MINCOST(0) & = 0 & \\
MINCOST(i) & = MINCOST(i-1)+r\cdot s_i & \textrm{ for } i < 4
\end{align}
Based on these equations, our algorithm works as follows.

We use an array \texttt{MINCOST[0..n]} to store $MINCOST(i)$ values, and use a two-dimensional array \texttt{OPT[1..n][1..n]} to store optimal schedules. 
\\\tt
\\
MINCOST[0] = 0\\
for i from 1 to min\{n, 3\} do\\
\mbox{\hspace{2em}}MINCOST[i] = MINCOST[i-1] + $r\cdot s_i$\\
\mbox{\hspace{2em}}for j from 1 to i-1 do\\
\mbox{\hspace{4em}}OPT[i][j] = OPT[i-1][j]\\
\mbox{\hspace{2em}}end\\
\mbox{\hspace{2em}}OPT[i][i] = `A'\\
end\\
for i from 4 to n do\\
\mbox{\hspace{2em}}MINCOST[i] = min\{MINCOST[i-1]+$r\cdot s_i$, MINCOST[i-4]+$4c$\}\\
\mbox{\hspace{2em}}if MINCOST[i-1]+$r\cdot s_i$ < MINCOST[i-4]+$4c$ then\\
\mbox{\hspace{4em}}for j from 1 to i-1 do\\
\mbox{\hspace{6em}}OPT[i][j] = OPT[i-1][j]\\
\mbox{\hspace{4em}}end\\
\mbox{\hspace{4em}}OPT[i][i] = `A'\\
\mbox{\hspace{2em}}else\\
\mbox{\hspace{4em}}for j from 1 to i-4 do\\
\mbox{\hspace{6em}}OPT[i][j] = OPT[i-4][j]\\
\mbox{\hspace{4em}}end\\
\mbox{\hspace{4em}}OPT[i][i-3] = OPT[i][i-2] = OPT[i][i-1] = OPT[i][i] = `B'\\
\mbox{\hspace{2em}}end\\
end\\
\rm

The final answer is stored in \texttt{OPT[n][1..n]}, with \texttt{OPT[n][i]} denoting the company chosen for the $i$-th week.

It takes constant time to get each \texttt{MINCOST[i]} value, and $i$ operations to store the optimal schedule in \texttt{OPT[i][1..i]}, so the overall time complexity is $O(n^2)$.

\section*{Chapter 6, Problem 15}
On most clear days, a group of your friends in the Astronomy Department
gets together to plan out the astronomical events they're going to try
observing that night. We'll make the following assumptions about the
events.
\begin{itemize}
\item There are $n$ events, which for simplicity we'll assume occur in sequence
separated by exactly one minute each. Thus event $j$ occurs
at minute $j$; if they don't observe this event at exactly minute $j$, then
they miss out on it.
\item The sky is mapped according to a one-dimensional coordinate system
(measured in degrees from some central baseline); event $j$ will be
taking place at coordinate $d_j$, for some integer value $d_j$. The telescope
starts at coordinate 0 at minute 0.
\item The last event, $n$, is much more important than the others; so it is
required that they observe event $n$.
\end{itemize}

The Astronomy Department operates a large telescope that can be
used for viewing these events. Because it is such a complex instrument, it
can only move at a rate of one degree per minute. Thus they do not expect
to be able to observe all $n$ events; they just want to observe as many as
possible, limited by the operation of the telescope and the requirement
that event $n$ must be observed.

We say that a subset $S$ of the events is \emph{viewable} if it is possible to
observe each event $j \in S$ at its appointed time $j$, and the telescope has
adequate time (moving at its maximum of one degree per minute) to move
between consecutive events in S.

\textbf{The problem.} Given the coordinates of each of the $n$ events, find a
viewable subset of maximum size, subject to the requirement that it
should contain event $n$. Such a solution will be called \emph{optimal}.

\textbf{Example.} Suppose the one-dimensional coordinates of the events are as
shown here.
\begin{table}[!h]
\begin{center}
\begin{tabular}{lrrrrrrrrr}
\hline
Event & 1 & 2 & 3 & 4 & 5 & 6 & 7 & 8 & 9\\
Coordinate & 1 & -4 & -1 & 4 & 5 & -4 & 6 & 7 & -2\\
\hline
\end{tabular}
\end{center}
\end{table}

Then the optimal solution is to observe events 1, 3, 6, 9. Note that the
telescope has time to move from one event in this set to the next, even
moving at one degree per minute.
\begin{itemize}
\item[(a)] Show that the following algorithm does not correctly solve this
problem, by giving an instance on which it does not return the correct
answer.
\begin{table}[!h]
\begin{center}
\begin{tabular}{l}
\hline
Mark all events $j$ with $|d_n-d_j| > n-j$ as illegal (as\\
\mbox{\hspace{2em}}observing them would prevent you from observing event $n$)\\
Mark all other events as legal\\
Initialize current position to coordinate 0 at minute 0\\
While not at end of event sequence\\
\mbox{\hspace{2em}}Find the earliest legal event $j$ that can be reached without\\
\mbox{\hspace{4em}}exceeding the maximum movement rate of the telescope\\
\mbox{\hspace{2em}}Add $j$ to the set $S$\\
\mbox{\hspace{2em}}Update current position to be coord.$d_j$ at minute $j$\\
Endwhile\\
Output the set $S$\\
\hline
\end{tabular}
\end{center}
\end{table}

In your example, say what the correct answer is and also what
the algorithm above finds.
\item[(b)] Give an efficient algorithm that takes values for the coordinates
$d_1, d_2, ..., d_n$ of the events and returns the \emph{size} of an optimal solution.
\end{itemize}

\section*{Answer}
\subsection*{Subproblem (a)}
Consider the following set of events:
\begin{table}[!h]
\begin{center}
\begin{tabular}{lrrrrrrrrr}
\hline
Event & 1 & 2 & 3 & 4 & 5\\
Coordinate & 0 & -1 & 2 & 3 & 2\\
\hline
\end{tabular}
\end{center}
\end{table}

The algorithm will mark all the five events as legal and add event 1, 2, and 5 in order to the set $S$, yielding the answer \{1, 2, 5\}, but the correct answer should be \{1, 3, 4, 5\}.

\subsection*{Subproblem (b)}
We use $OPT(i)$ to denote the size of the maximum viewable subset of the first $i$ events that contains event $i$. It can be achieved through observing some previous event such that the telescope can catch event $i$ in time thereafter:
\begin{align}
\nonumber OPT(i) & = & \max_{1\le j < i, |d_i-d_j|\le i-j}{OPT(j)}+1 & \textrm{\hspace{2em} if }\{j\mid |d_i-d_j|\le i-j\}\ne \emptyset;\\\nonumber
 & & 1 & \textrm{\hspace{2em} else if }|d_i|\le i;\\
 & & 0 & \textrm{\hspace{2em} otherwise}
\end{align}

Based on this equation, we use an array \texttt{OPT[1..n]} to store the values of $OPT(i)$ and design the algorithm as follows.
\\
\\\tt
for i from 1 to n do\\
\mbox{\hspace{2em}}if $|d_i| > i$ then\\
\mbox{\hspace{4em}}OPT[i] = 0\\
\mbox{\hspace{2em}}else\\
\mbox{\hspace{4em}}OPT[i] = 1\\
\mbox{\hspace{4em}}for j from 1 to i-1 do\\
\mbox{\hspace{5em}}if OPT[j]$\ne$0 and $|d_i-d_j|\le i-j$ and OPT[j]+1>OPT[i] then\\
\mbox{\hspace{7em}}OPT[i] = OPT[j]+1\\
\mbox{\hspace{5em}}end\\
\mbox{\hspace{4em}}end\\
\mbox{\hspace{2em}}end\\
end\\
\rm

The final answer is \texttt{OPT[n]}.

Since filling each cell in \texttt{OPT[1..n]} takes $O(i)$ time, the time complexity of the algorithm is $O(n^2)$.
\end{document}
