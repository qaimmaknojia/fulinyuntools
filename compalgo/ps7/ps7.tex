\documentclass[12pt,letterpaper]{article}
\usepackage[latin1]{inputenc}
\usepackage{amsmath}
\usepackage{amsfonts}
\usepackage{amssymb}
\usepackage{graphicx}
\usepackage{amsthm}

\newtheorem*{eg}{Example}

\author{Linyun Fu}
\title{CSCI 4020 Computer Algorithms Spring 2011\\
Problem Set 7}
\begin{document}
\maketitle
\section*{Chapter 7, Problem 29}
Some of your friends have recently graduated and started a small company,
which they are currently running out of their parents' garages in
Santa Clara. They're in the process of porting all their software from an
old system to a new, revved-up system; and they're facing the following
problem.

They have a collection of $n$ software applications, $\{1, 2, \dots, n\}$, running
on their old system; and they'd like to port some of these to the new
system. If they move application $i$ to the new system, they expect a net
(monetary) benefit of $b_i \ge 0$. The different software applications interact
with one another; if applications $i$ and $j$ have extensive interaction, then
the company, will incur an expense if they move one of $i$ or $j$ to the new
system but not both; let's denote this expense by $x_{ij} \ge 0$.

So, if the situation were really this simple, your friends would just
port all $n$ applications, achieving a total benefit of $\sum_i b_i$. Unfortunately,
there's a problem, ...

Due to small but fundamental incompatibilities between the two
systems, there's no way to port application 1 to the new system; it will
have to remain on the old system. Nevertheless, it might still pay off to
port some of the other applications, accruing the associated benefit and
incurring the expense of the interaction between applications on different
systems.

So this is the question they pose to you: Which of the remaining
applications, if any, should be moved? Give a polynomial-time algorithm
to find a set $S \subseteq \{2, 3, \dots, n\}$ for which the sum of the benefits minus the
expenses of moving the applications in $S$ to the new system is maximized.

\section*{Answer}
We define the value of a porting plan $S$, namely $q(S)$, to be the sum of the benefits minus the
expenses of moving the applications in $S$ to the new system.

We solve this problem by reducing it to the Minimum-Cut problem.

Given applications 1 through $n$, each with a net benefit $b_i \ge 0$, and each pair with a separation expense $x_{ij} \ge 0$, we create a flow network as follows.
\begin{itemize}
\item For each application $i$, we create a node $i$ in the flow network. We also create a source node $s$ and a sink node $t$;
\item We create an edge from $s$ to 1 with capacity $+\infty$;
\item We create an edge from each node $i$ to $t$ with capacity $b_i$;
\item We create two edges for each node pair $i$ and $j$, one from $i$ to $j$, the other from $j$ to $i$, both with capacity $x_{ij}$.
\end{itemize}

Then we run the Min-Cut algorithm on the flow network. Suppose we find a minimum cut ($\{s\}\cup A, \{t\}\cup S$), then the applications in $S$ should should be ported to the new system, and the rest should remain on the old system. Figure~1 illustrates the conversion.
\begin{figure}
\begin{center}
\includegraphics[width=0.7\textwidth]{7.29.eps}
\caption{Convert the problem to the Minimum-Cut problem}
\end{center}
\end{figure}

Now we prove that each porting plan $S$ of value $q(S)$ corresponds to a cut ($\{s\}\cup A, \{t\}\cup S$) of size $q'(S)=\sum_i b_i-q(S)$, and vice versa.
\begin{itemize}
\item Given a porting plan, for each application $i$ that is ported to the new system, we let node $i$ in set $S$ in the flow network; for each of the rest applications, we let its corresponding node in $A$. The value of this porting plan is 
\begin{align}
q(S)=\sum_{i\in S}b_i-\sum_{|S\cap \{i,j\}|=1}x_{ij}
\end{align}
and the size of the cut is 
\begin{align}
q'(S)=\sum_{i\notin S}b_i+\sum_{|S\cap \{i,j\}|=1}x_{ij}
\end{align}
so $q(S)+q'(S)=\sum_i b_i$.
\item Given a cut ($\{s\}\cup A, \{t\}\cup S$) with a finite size, we know that node 1 must not belong to $S$, we port application $i$ to the new system if node $i$ belongs to $S$ and let the rest applications remain on the old system. We also get $q(S)+q'(S)=\sum_i b_i$ through the same calculation as above.
\end{itemize}
Therefore the porting plan with the maximum value corresponds to the cut with the minimum value.

Constructing the flow network consists of creating $n+2$ nodes and $O(n^2)$ edges with capacities, which can be completed in polynomial time. Since the Minimum-Cut problem can be solved in polynomial time, the overall time complexity is polynomial.

\section*{Chapter 7, Problem 30}
Consider a variation on the previous problem. In the new scenario, any
application can potentially be moved, but now some of the benefits $b_i$ for
moving to the new system are in fact negative: If $b_i < 0$, then it is preferable
(by an amount quantified in $b_i$) to keep $i$ on the old system. Again, give
a polynomial-time algorithm to find a set $S \subseteq \{1, 2, \dots, n\}$ for which the
sum of the benefits minus the expenses of moving the applications in $S$
to the new system is maximized.

\section*{Answer}
We define the value of a porting plan $S$, namely $q(S)$, to be the sum of the benefits minus the
expenses of moving the applications in $S$ to the new system.

We solve this problem by reducing it to the Minimum-Cut problem.

Given applications 1 through $n$, each with a net benefit $b_i$, and each pair with a separation expense $x_{ij} \ge 0$, we create a flow network as follows.
\begin{itemize}
\item For each application $i$, we create a node $i$ in the flow network. We also create a source node $s$ and a sink node $t$;
\item We create an edge from $s$ to each $i$ satisfying $b_i<0$ with capacity $-b_i$;
\item We create an edge from each $i$ satisfying $b_i>0$ to $t$ with capacity $b_i$;
\item We create two edges for each node pair $i$ and $j$, one from $i$ to $j$, the other from $j$ to $i$, both with capacity $x_{ij}$.
\end{itemize}

Then we run the Min-Cut algorithm on the flow network. Suppose we find a minimum cut ($\{s\}\cup A, \{t\}\cup S$), then the applications in $S$ should should be ported to the new system, and the rest should remain on the old system. Figure~2 illustrates the conversion.
\begin{figure}
\begin{center}
\includegraphics[width=0.7\textwidth]{7.30.eps}
\caption{Convert the problem to the Minimum-Cut problem, $b_i>0, b_j<0$}
\end{center}
\end{figure}

Now we prove that each porting plan $S$ of value $q(S)$ corresponds to a cut ($\{s\}\cup A, \{t\}\cup S$) of size $q'(S)=\sum_{b_i>0} b_i-q(S)$, and vice versa.
\begin{itemize}
\item Given a porting plan, for each application $i$ that is ported to the new system, we let its corresponding node in set $S$ in the flow network; for each of the rest applications, we let its corresponding node in $A$. The value of this porting plan is 
\begin{align}
q(S)=\sum_{i\in S}b_i-\sum_{|S\cap \{i,j\}|=1}x_{ij}
\end{align}
and the size of the cut is 
\begin{align}
q'(S)=\sum_{i\notin S\textrm{ and }b_i>0}b_i+\sum_{i\in S\textrm{ and }b_i<0}-b_i+\sum_{|S\cap \{i,j\}|=1}x_{ij}
\end{align}
so 
\begin{align}
\nonumber q(S)+q'(S) & = \sum_{i\in S}b_i+\sum_{i\notin S\textrm{ and }b_i>0}b_i-\sum_{i\in S\textrm{ and }b_i<0}b_i\\\nonumber
& = \sum_{i\in S\textrm{ and }b_i>0}b_i+\sum_{i\in S\textrm{ and }b_i<0}b_i+\sum_{i\notin S\textrm{ and }b_i>0}b_i-\sum_{i\in S\textrm{ and }b_i<0}b_i\\\nonumber
& = \sum_{b_i>0}b_i
\end{align}
\item Given a cut ($\{s\}\cup A, \{t\}\cup S$), we port application $i$ to the new system if node $i$ belongs to $S$ and let the rest applications remain on the old system. We also get $q(S)+q'(S)=\sum_{b_i>0} b_i$ through the same calculation as above.
\end{itemize}
Therefore the porting plan with the maximum value corresponds to the cut with the minimum value.

Constructing the flow network consists of creating $n+2$ nodes and $O(n^2)$ edges with capacities, which can be completed in polynomial time. Since the Minimum-Cut problem can be solved in polynomial time, the overall time complexity is polynomial.

\end{document}
