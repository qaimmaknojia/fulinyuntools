\documentclass[12pt,letterpaper]{article}
\usepackage[latin1]{inputenc}
\usepackage{amsmath}
\usepackage{amsfonts}
\usepackage{amssymb}
\usepackage{graphicx}
\usepackage{amsthm}

\newtheorem*{eg}{Example}

\author{Linyun Fu}
\title{CSCI 4020 Computer Algorithms Spring 2011\\
Problem Set 6}
\begin{document}
\maketitle
\section*{Chapter 7, Problem 13}
in a standard $s$-$t$ Maximum-Flow Problem, we assume edges have capacities,
and there is no limit on how much flow is allowed to pass through a node. In this problem, we consider the variant of the Maximum-Flow and
Minimum-Cut problems with node capacities.

Let $G = (V, E)$ be a directed graph, with source $s \in V$; sink $t \in V$, and
nonnegative node capacities $\{c_v \ge 0\}$ for each $v \in V$. Given a flow $f$ in this
graph, the flow though a node $v$ is defined as $f^{\textrm{in}}(v)$. We say that a flow
is feasible if it satisfies the usual flow-conservation constraints and the
node-capacity constraints: $f^{\textrm{in}}(v)\le c_v$ for all nodes.

Give a polynomial-time algorithm to find an $s$-$t$ maximum flow in
such a node-capacitated network. Define an $s$-$t$ cut for node-capacitated
networks, and show that the analogue of the Max-Flow Min-Cut Theorem
holds true.

\section*{Answer}
\begin{figure}
\begin{center}
%\includegraphics[width=0.7\textwidth]{7.13.eps}
\caption{Reducing the problem to a normal maximum flow problem}
\end{center}
\end{figure}

\section*{Chapter 7, Problem 28}
A group of students has decided to add some features to Cornell's on-line
Course Management System (CMS), to handle aspects of course planning
that are not currently covered by the software. They're beginning with a
module that helps schedule office hours at the start of the semester.

Their initial prototype works as follows. The office hour schedule will
be the same from one week to the next, so it's enough to focus on the
scheduling problem for a single week. The course administrator enters
a collection of nonoverlapping one-hour time intervals $I_1, I_2, \dots, I_k$ when
it would be possible for teaching assistants (TAs) to hold office hours;
the eventual office-hour schedule will consist of a subset of some, but
generally, not all, of these time slots. Then each of the TAs enters his or
her weekly schedule, showing the times when he or she would be available
to hold office hours.

Finally, the course administrator specifies, for parameters $a$, $b$, and
$c$, that they would like each TA to hold between $a$ and $b$ office hours per
week, and they would like a total of exactly $c$ office hours to be held over
the course of the week.

The problem, then, is how to assign each TA to some of the office hour
time slots, so that each TA is available for each of his or her office hour
slots, and so that the right number of office hours gets held. (There
should be only one TA at each office hour.)

{\eg \rm Suppose there are five possible time slots for office hours:}
\begin{quote}
{\it $I_1$ = Mon 3 -- 4 P.M.; $I_2$ = Tue 1 -- 2 P.M.; $I_3$ = Wed 10 -- 11 A.M.; $I_4$ = Wed 3 -- 4 P.M.; and $I_5$ = Thu 10 -- 11 A.M..}
\end{quote}
There are two TAs; the first would be able to hold office hours at any
time on Monday or Wednesday afternoons, and the second would be able
to hold office hours at any time on Tuesday, Wednesday, or Thursday.
(In general, TA availability might be more complicated to specify than
this, but we're keeping this example simple.) Finally, each TA should hold
between $a = 1$ and $b = 2$ office hours, and we want exactly $c = 3$ office hours
per week total.

One possible solution would be to have the first TA hold office hours
in time slot $I_1$, and the second TA to hold office hours In time slots $I_2$
and $I_5$.
\begin{itemize}
\item[(a)] Give a polynomial-time algorithm that takes the Input to an Instance
of this problem (the time slots, the TA schedules, and the parameters
a, b, and c) and does one of the following two things:
\begin{itemize}
\item Constructs a valid schedule for office hours, specifying which
TA will cover which time slots, or
\item Reports (correctly) that there is no valid way to schedule office
hours.
\end{itemize}
\item[(b)] This office-hour scheduling feature becomes very popular, and so
course staffs begin to demand more. In particular, they observe that
it's good to have a greater density of office hours closer to the due
date of a homework assignment.

So what they want to be able to do is to specify an office-hour
density parameter for each day of the week: The number $d_i$ specifies
that they want to have at least $d_i$ office hours on a given day $i$ of the
week.

For example, suppose that in our previous example, we add the
constraint that we want at least one office hour on Wednesday and at
least one office hour on Thursday. Then the previous solution does
not work; but there is a possible solution in which we have the first
TA hold office hours in time slot $I_1$, and the second TA hold office
hours In time slots $I_3$ and $I_5$. (Another solution would be to have the
first TA hold office hours in time slots $I_1$ and $I_4$, and the second TA
hold office hours in time slot $I_5$.)

Give a polynomial-time algorithm that computes office-hour
schedules under this more complex set of constraints. The algorithm
should either construct a schedule or report (correctly) that
none exists.
\end{itemize}

\section*{Answer}
\subsection*{Problem (a)}
\begin{figure}
\begin{center}
%\includegraphics[width=0.7\textwidth]{7.28a.eps}
\caption{Reducing problem (a) into a circulation problem, all nodes have demand 0}
\end{center}
\end{figure}

\subsection*{Problem (b)}
\begin{figure}
\begin{center}
%\includegraphics[width=0.7\textwidth]{7.28b.eps}
\caption{Reducing problem (b) into a circulation problem, all nodes have demand 0}
\end{center}
\end{figure}

\end{document}
