\documentclass[12pt,letterpaper]{article}
\usepackage[latin1]{inputenc}
\usepackage{amsmath}
\usepackage{amsfonts}
\usepackage{amssymb}
\usepackage{graphicx}
\usepackage{amsthm}

\newtheorem*{eg}{Example}

\author{Linyun Fu}
\title{CSCI 4020 Computer Algorithms Spring 2011\\
Problem Set 6}
\begin{document}
\maketitle
\section*{Chapter 7, Problem 13}
in a standard $s$-$t$ Maximum-Flow Problem, we assume edges have capacities,
and there is no limit on how much flow is allowed to pass through a node. In this problem, we consider the variant of the Maximum-Flow and
Minimum-Cut problems with node capacities.

Let $G = (V, E)$ be a directed graph, with source $s \in V$; sink $t \in V$, and
nonnegative node capacities $\{c_v \ge 0\}$ for each $v \in V$. Given a flow $f$ in this
graph, the flow though a node $v$ is defined as $f^{\textrm{in}}(v)$. We say that a flow
is feasible if it satisfies the usual flow-conservation constraints and the
node-capacity constraints: $f^{\textrm{in}}(v)\le c_v$ for all nodes.

Give a polynomial-time algorithm to find an $s$-$t$ maximum flow in
such a node-capacitated network. Define an $s$-$t$ cut for node-capacitated
networks, and show that the analogue of the Max-Flow Min-Cut Theorem
holds true.

\section*{Answer}
To find an $s$-$t$ maximum flow in a node-capacitated network $G$, we convert the problem to a standard $s$-$t$ problem, i.e., given $G$, we create the standard flow network $G'$ as follows.
\begin{itemize}
\item For each node $v\in V$, we create two nodes, namely $v_{in}$ and $v_{out}$, and create an edge from $v_{in}$ to $v_{out}$ with capacity $c_v$ in $G'$;
\item We let $s_{in}$ be the source in $G'$, and $t_{out}$ the sink;
\item For each edge from $v_i$ to $v_j$ in $G$, we create an edge from $v_{iout}$ to $v_{jin}$ with capacity $+\infty$ in $G'$.
\end{itemize}
Then we run some maximum flow algorithm on $G'$, and let the flow on the edge from $v_i$ to $v_j$ in $G$ be the same as the flow on the edge $v_{iout}$ to $v_{jin}$ in $G'$. This is the maximum flow on $G$. Figure~1 gives an example of the conversion.
\begin{figure}
\begin{center}
\includegraphics[width=0.7\textwidth]{7.13.eps}
\caption{Reducing the problem to a normal maximum flow problem}
\end{center}
\end{figure}

Next we prove that a flow in $G$ corresponds to a flow of the same size in $G'$, and vice versa.
\begin{itemize}
\item Given a flow in $G$, we know that for each $v\in V$, $f^{in}v\le c_v$, so we can assign a flow of size $f^{in}v$ on the edge from $v_{in}$ to $v_{out}$ in $G'$. We also assign the edge from $v_{iout}$ to $v_{jin}$ in $G'$ a flow of the same size as the edge from $v_i$ to $v_j$ in $G$. Since $f^{out}s$ in $G$ is the same as $f^{out}s_{out}$ in $G'$, which is the same as $f^{out}s_{in}$, we have created a flow with the same size in $G'$.
\item Given a flow in $G'$, we know that the flow on each edge from $v_{in}$ to $v_{out}$ in $G'$ is less than $c_v$, thus $f^{in}v_{in}$ is less than $c_v$, so we can let edge from $v_i$ to $v_j$ in $G$ carry flow of the same size as the edge from $v_{iout}$ to $v_{jin}$ in $G'$. Then $f^{out}s$ in $G$ is the same as $f^{out}s_{out}$ in $G'$, which is the same as $f^{out}s_{in}$, we have created a flow with the same size in $G$.
\end{itemize}
Therefore a maximum flow in $G'$ corresponds to a maximum flow in $G$ and they have the same size.

Constructing $G'$ consists of creating $2n$ nodes, where $n = |V|$ in $G$, and assigning capacities to $m+n$ edges, where $m = |E|$ in G, which can be finished in polynomial time. Since finding the maximum flow for a standard flow network takes polynomial time, the overall time complexity is polynomial.

An $s$-$t$ cut in $G$ can be defined as the set of nodes that separate $s$ from $t$ if deleted. The size of a cut is the sum of node capacity in the cut. Note that the minimum cut in $G'$ must be a set of edges each from $v_{in}$ to $v_{out}$, which corresponds to a set of nodes in $G$, i.e., edge ($v_{in}$, $v_{out}$) corresponds to node $v$. Also the minimum cut in $G$, which is a set of nodes, corresponds to a set of edges ($v_{in}$, $v_{out}$) in $G'$, so the Max-Flow Min-Cut Theorem holds true for node-capacitated flow networks.

\section*{Chapter 7, Problem 28}
A group of students has decided to add some features to Cornell's on-line
Course Management System (CMS), to handle aspects of course planning
that are not currently covered by the software. They're beginning with a
module that helps schedule office hours at the start of the semester.

Their initial prototype works as follows. The office hour schedule will
be the same from one week to the next, so it's enough to focus on the
scheduling problem for a single week. The course administrator enters
a collection of nonoverlapping one-hour time intervals $I_1, I_2, \dots, I_k$ when
it would be possible for teaching assistants (TAs) to hold office hours;
the eventual office-hour schedule will consist of a subset of some, but
generally, not all, of these time slots. Then each of the TAs enters his or
her weekly schedule, showing the times when he or she would be available
to hold office hours.

Finally, the course administrator specifies, for parameters $a$, $b$, and
$c$, that they would like each TA to hold between $a$ and $b$ office hours per
week, and they would like a total of exactly $c$ office hours to be held over
the course of the week.

The problem, then, is how to assign each TA to some of the office hour
time slots, so that each TA is available for each of his or her office hour
slots, and so that the right number of office hours gets held. (There
should be only one TA at each office hour.)

{\eg \rm Suppose there are five possible time slots for office hours:}
\begin{quote}
{\it $I_1$ = Mon 3 -- 4 P.M.; $I_2$ = Tue 1 -- 2 P.M.; $I_3$ = Wed 10 -- 11 A.M.; $I_4$ = Wed 3 -- 4 P.M.; and $I_5$ = Thu 10 -- 11 A.M..}
\end{quote}
There are two TAs; the first would be able to hold office hours at any
time on Monday or Wednesday afternoons, and the second would be able
to hold office hours at any time on Tuesday, Wednesday, or Thursday.
(In general, TA availability might be more complicated to specify than
this, but we're keeping this example simple.) Finally, each TA should hold
between $a = 1$ and $b = 2$ office hours, and we want exactly $c = 3$ office hours
per week total.

One possible solution would be to have the first TA hold office hours
in time slot $I_1$, and the second TA to hold office hours In time slots $I_2$
and $I_5$.
\begin{itemize}
\item[(a)] Give a polynomial-time algorithm that takes the Input to an Instance
of this problem (the time slots, the TA schedules, and the parameters
a, b, and c) and does one of the following two things:
\begin{itemize}
\item Constructs a valid schedule for office hours, specifying which
TA will cover which time slots, or
\item Reports (correctly) that there is no valid way to schedule office
hours.
\end{itemize}
\item[(b)] This office-hour scheduling feature becomes very popular, and so
course staffs begin to demand more. In particular, they observe that
it's good to have a greater density of office hours closer to the due
date of a homework assignment.

So what they want to be able to do is to specify an office-hour
density parameter for each day of the week: The number $d_i$ specifies
that they want to have at least $d_i$ office hours on a given day $i$ of the
week.

For example, suppose that in our previous example, we add the
constraint that we want at least one office hour on Wednesday and at
least one office hour on Thursday. Then the previous solution does
not work; but there is a possible solution in which we have the first
TA hold office hours in time slot $I_1$, and the second TA hold office
hours In time slots $I_3$ and $I_5$. (Another solution would be to have the
first TA hold office hours in time slots $I_1$ and $I_4$, and the second TA
hold office hours in time slot $I_5$.)

Give a polynomial-time algorithm that computes office-hour
schedules under this more complex set of constraints. The algorithm
should either construct a schedule or report (correctly) that
none exists.
\end{itemize}

\section*{Answer}
\subsection*{Problem (a)}
We convert this problem into a circulation problem as follows.
\begin{itemize}
\item 
\end{itemize}
Then we solve the circulation problem with some network flow algorithm, if we cannot find a feasible circulation for $G$, we know that a schedule does not exist, otherwise all the edges from $TA_i$ to $I_j$ with flow of size 1 corresponds to a schedule, i.e., if the edge from $TA_i$ to $I_j$ carries flow of size 1, TA $i$ should work during time interval $I_j$. Figure~2 illustrates the conversion.
\begin{figure}
\begin{center}
%\includegraphics[width=0.7\textwidth]{7.28a.eps}
\caption{Reducing problem (a) into a circulation problem, all nodes have demand 0}
\end{center}
\end{figure}

Next we prove that a solution to the original problem corresponds to a feasible circulation in $G$, and vice versa.
\begin{itemize}
\item Given a solution to the original problem, 
\item Given a feasible circulation in $G$
\end{itemize}

\subsection*{Problem (b)}
\begin{figure}
\begin{center}
%\includegraphics[width=0.7\textwidth]{7.28b.eps}
\caption{Reducing problem (b) into a circulation problem, all nodes have demand 0}
\end{center}
\end{figure}

\end{document}
