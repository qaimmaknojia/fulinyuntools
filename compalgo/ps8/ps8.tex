\documentclass[12pt,letterpaper]{article}
\usepackage[latin1]{inputenc}
\usepackage{amsmath}
\usepackage{amsfonts}
\usepackage{amssymb}
\usepackage{graphicx}
\usepackage{amsthm}

\newtheorem*{eg}{Example}

\author{Linyun Fu}
\title{CSCI 4020 Computer Algorithms Spring 2011\\
Problem Set 8}
\begin{document}
\maketitle
\section*{Chapter 8, Problem 3}
Suppose you're helping to organize a summer sports camp, and the
following problem comes up. The camp is supposed to have at least
one counselor who's skilled at each of the $n$ sports covered by the camp
(baseball, volleyball, and so on). They have received job applications from
$m$ potential counselors. For each of the $n$ sports, there is some subset
of the $m$ applicants qualified in that sport. The question is: For a given
number $k < m$, is it possible to hire at most $k$ of the counselors and have
at least one counselor qualified in each of the $n$ sports? We'll call this the
{\em Efficient Recruiting Problem}.

Show that Efficient Recruiting is NP-complete.

\section*{Answer}
Given the subset of applicants qualified in each sport, we can easily know the subset of sports each applicant is skilled at. We denote an instance of Efficient Recruiting as $(U, S_1, S_2, \dots, S_m, k)$, where $U$ is the full set of the $n$ sports, $S_i$ are the subset of sports applicant $i$ is skilled at, and $k$ is the largest number of counselors to hire.

First, given the subset of sports each applicant is skilled at, and no more than $k$ applicants as a certificate, we can check whether these applicants cover all the $n$ sports by calculating the union of the set of sports each of these $k$ applicants is skilled at and see if it equals the full set of all the sports. This calculation takes polynomial time, therefore Efficient Recruiting is NP.

Next we show that Efficient Recruiting is harder than Set Cover. Given an instance of Set Cover $(U, S_1, S_2, \dots, S_m, k)$, we construct an instance of Efficient Recruiting $(U', S_1', S_2', \dots, S_{m'}', k')$ as follows.
\begin{itemize}
\item $U' = U$;
\item $m'=m$ and $S_i' = S_i$ for each $i$ from 1 to $m$;
\item $k'=k$.
\end{itemize}

Given the answer to a Set Cover problem instance, we know whether we can find at most $k$ sets from $S_1, S_2, \dots, S_m$ whose union is the full set $U$, so we know whether we can find at most $k'$ sets from $S_1', S_2', \dots, S_{m'}'$ whose union is $U'$, which means we can hire at most $k'$ counselors and have at least one counselor qualified in each sport.

On the other hand, given the answer to an Efficient Recruiting problem instance, we know whether we can find at most $k'$ counselors to cover all the sports, which means we can find at most $k'$ sets from $S_1', S_2', \dots, S_{m'}'$ whose union is $U'$, which means we can find at most $k$ sets from $S_1, S_2, \dots, S_m$ whose union is the full set $U$.

The reduction consists of creating the full set $U'$, $m$ subsets $S_i'$, and copying the value of $k$, which can be completed in polynomial time, so Efficient Recruiting is NP-complete.

\section*{Chapter 8, Problem 9}
Consider the following problem. You are managing a communication
network, modeled by a directed graph $G = (V, E)$. There are $c$ {\em users} who
are interested in malting use of this network. User $i$ (for each $i = 1, 2, \dots, c$)
issues a {\em request} to reserve a specific path $P_i$ in $G$ on which to transmit
data.

You are interested in accepting as many of these path requests as
possible, subject to the following restriction: if you accept both $P_i$ and $P_j$,
then $P_i$ and $P_j$ cannot share any nodes.

Thus, the {\em Path Selection Problem} asks: Given a directed graph $G =
(V, E)$, a set of requests $P_1, P_2, \dots, P_c$ -- each of which must be a path in
$G$ -- and a number $k$, is it possible to select at least $k$ of the paths so that
no two of the selected paths share any nodes?

Prove that Path Selection is NP-complete.

\section*{Answer}
We denote an instance of Path Selection as $(G=(V,E), P_1, P_2, \dots, P_c, k)$.

First, given $G$ and no more than $k$ paths from $P_1, P_2, \dots, P_c$ as a certificate, we can check whether these paths share any node by calculating the intersection of the node set of each path in polynomial time, therefore Path Selection is NP.

Next we show that Path Selection is harder than Independent Set. Given an instance of Independent Set $(G=(V,E), k)$, we construct an instance of Path Selection $(G'=(V',E'), P_1, P_2, \dots, P_c, k')$ as follows.
\begin{itemize}
\item $V' = E\cup V$, i.e., for each edge $(v_i, v_j)\in E$, we create a node $v_{ij}$ in $V'$, and for each node $v_i\in V$, we create a node $v_i$ in $V'$;
\item $c = |V|$;
\item For each $i$, we connect all nodes of form $v_{ij}$ and $v_{ji}$ and $v_i$ in some order to form a chain, and we let $P_i$ be this chain;
\item $k'=k$.
\end{itemize}
Figure~1 illustrate the reduction.
\begin{figure}[h]
\begin{center}
\includegraphics[width=0.7\textwidth]{8.9.eps}
\caption{Reduce Independent Set to Path Selection, $P_1=(v_1, v_{12}, v_{13}, v_{14})$, $P_2=(v_2, v_{12}, v_{23})$, $P_3=(v_3, v_{13}, v_{23})$, $P_4=(v_{14}, v_4)$}
\end{center}
\end{figure}

Given the answer to a Independent Set problem instance, we know whether we can find at least $k$ nodes from $V$ where no two nodes are adjacent, so we know whether we can find at least $k'$ paths from $P_1, P_2, \dots, P_c$ where no two paths share any nodes, because $v_i$ and $v_j$ are adjacent in $G$ iff $P_i$ and $P_j$ share node $v_{ij}$ in $G'$.

On the other hand, given the answer to a Path Selection problem instance, we know whether we can find at least $k'$ paths where no two paths share any nodes, which means we can find at least $k$ nodes from $V$ where no two nodes are adjacent.

The reduction consists of creating $|E|+|V|$ nodes and $O(|V|^2)$ edges, setting $|V|$ paths and copying $k$, which can be completed in polynomial time, so Path Selection is NP-complete.

\end{document}

