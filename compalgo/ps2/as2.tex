\documentclass[12pt,letterpaper]{article}
\usepackage[latin1]{inputenc}
\usepackage{amsmath}
\usepackage{amsfonts}
\usepackage{amssymb}
\usepackage{graphicx}

\newtheorem{clm}{Claim}

\author{Linyun Fu}
\title{CSCI 4020 Computer Algorithms Spring 2011\\
Problem Set 2}
\begin{document}
\maketitle
\section*{Chapter 4, Problem 20}
Every September, somewhere In a far-away mountainous part of the
world, the county highway crews get together and decide which roads to
keep clear through the coming winter. There are $n$ towns in this county,
and the road system can be viewed as a (connected) graph $G = (V, E)$ on
this set of towns, each edge representing a road joining two of them.
In the winter, people are high enough up in the mountains that they
stop worrying about the length of roads and start worrying about their
altitude -- this is really what determines how difficult the trip will be.

So each road -- each edge $e$ in the graph -- is annotated with a number
$a_e$ that gives the altitude of the highest point on the road. We'll assume
that no two edges have exactly the same altitude value $a_e$. The \emph{height} of
a path $P$ in the graph is then the maximum of $a_e$ over all edges $e$ on $P$.
Finally, a path between towns $i$ and $j$ is declared to be winter-optimal if it
achieves the minimum possible height over all paths from $i$ to $j$.

The highway crews are going to select a set $E' \subseteq E$ of the roads to keep
clear through the winter; the rest will be left unmaintained and kept off
limits to travelers. They all agree that whichever subset of roads $E'$ they
decide to keep clear, it should have the property that $(V, E')$ is a connected
subgraph; and more strongly, for every pair of towns $i$ and $j$, the height
of the winter-optimal path in $(V, E')$ should be no greater than it is in the
full graph $G = (V, E)$. We'll say that $(V, E')$ is a \emph{minimum-altitude connected
subgraph} if it has this property.

Given that they're going to maintain this key property, however, they
otherwise want to keep as few roads clear as possible. One year, they hit
upon the following conjecture:
\begin{quote}
The minimum spanning tree of $G$, with respect to the edge weights $a_e$, is a
minimum-altitude connected subgraph.
\end{quote}
(In an earlier problem, we claimed that there is a unique minimum spanning
tree when the edge weights are distinct. Thus, thanks to the assumption
that all $a_e$ are distinct, it is okay for us to speak of the minimum
spanning tree.)

Initially, this conjecture is somewhat counterintuitive, since the minimum
spanning tree is trying to minimize the sum of the values $a_e$, while
the goal of minimizing altitude seems to be asking for a fully different
thing. But lacking an argument to the contrary, they begin considering an
even bolder second conjecture:
\begin{quote}
A subgraph $(V, E')$ is a minimum-altitude connected subgraph if and only if
it contains the edges of the minimum spanning tree.
\end{quote}
Note that this second conjecture would immediately imply the first one,
since a minimum spanning tree contains its own edges.

So here's the question.
\begin{itemize}
\item[(a)] Is the first conjecture true, for all choices of $G$ and distinct altitudes
$a_e$? Give a proof or a counterexample with explanation.
\item[(b)] Is the second conjecture true, for all choices of $G$ and distinct altitudes
$a_e$? Give a proof or a counterexample with explanation.
\end{itemize}

\section*{Answer}
We try to prove the following claim, which is similar to the \emph{Cut Property} taught in the class.
\begin{clm}
For any cut $(S, V-S)$ of V, the edge with the minimum altitude value across the cut must be an edge in all minimum-altitude connected subgraphs.
\end{clm}
Suppose edge $e=(u,v)$ is the edge with the minimum altitude value across cut $(S, V-S)$, and one minimum-altitude connected subgraph $(V,E')$ does not contain $e$. Suppose the winter-optimal path between $u$ and $v$ in $E'$ is $P$, as shown in Figure~1.

\begin{figure}
\begin{center}
\includegraphics[width=0.7\textwidth]{4.20.eps}
\caption{Path $P$ in $(V, E')$ has a greater altitude than edge $e$}
\end{center}
\vspace{-2ex}
\end{figure}

Then $P$ must contain at least one edge $e'$ that is across the cut. Since $e$ is the minimum-altitude edge across the cut, we have $a_{e'}>a_e$, and the height of $P$ is larger than or equal to $a_{e'}$, therefore larger than $a_e$, but the height of winter-optimal path in the full graph $(V,E)$ is less than or equal to $a_e$ because edge $e$ makes one path between $u$ and $v$. This means the height of winter-optimal path in $(V,E')$ is greater than it is in the full graph, which contradicts the definition of minimum-altitude connected subgraph. This completes the proof of Claim~1.

Since the minimum spanning tree is composed of edges with minimum altitude values across cuts, Claim~1 directly yields the following result.
\begin{clm}
Every minimum-altitude connected subgraph contains all the edges in the minimum spanning tree.
\end{clm}
And because adding edges to a minimum-altitude connected subgraph always yields another minimum-altitude connected subgraph, so we also have the following result.
\begin{clm}
Every supergraph of the minimum spanning tree is a minimum-altitude connected subgraph.
\end{clm}
It follows directly from Claim~2 and Claim~3 that both conjectures in the problem are true. That is, (a) the minimum spanning tree of $G$, with respect to the edge weights $a_e$, is a minimum-altitude connected subgraph, and (b) a subgraph $(V, E')$ is a minimum-altitude connected subgraph if and only if it contains the edges of the minimum spanning tree.

\section*{Chapter 6, Problem 7}
As a solved exercise in Chapter 5, we gave an algorithm with $O(n \log n)$
running time for the following problem. We're looking at the price of a
given stock over $n$ consecutive days, numbered $i = 1, 2, ..., n$. For each
day $i$, we have a price $p(i)$ per share for the stock on that day. (We'll
assume for simplicity that the price was fixed during each day.) We'd like
to know: How should we choose a day $i$ on which to buy the stock and a
later day $j > i$ on which to sell it, if we want to maximize the profit per
share, $p(j) - p(i)$? (If there is no way to make money during the $n$ days, we
should conclude this instead.)

In the solved exercise, we showed how to find the optimal pair of
days $i$ and $j$ in time $O(n \log n)$. But, in fact, it's possible to do better than
this. Show how to find the optimal numbers $i$ and $j$ in time $O(n)$.

\section*{Answer}
We define $MIN(i)$ to be the lowest price among the first $i$ days, $MINDAY(i)$ the day that has this lowest price, and $OPT(i)$ to be the greatest profit that can be made among the first $i$ days, which is achieved through buying the stock on day $BUY(i)$ and sell it on day $SELL(i)$. Since $OPT(i)$ is either achieved through selling the stock on the $i$-th day, or not selling it on that day, we have
\begin{align}
OPT(i) = \max\{p(i)-MIN(i), OPT(i-1)\}
\end{align}
Based on this equation, our algorithm works as follows.

First, we calculate $MIN(i)$ and $MINDAY(i)$ for $i$ from $1$ to $n$ and use two arrays \texttt{MIN[1..n]} and \texttt{MINDAY[1..n]} to store the values, i.e., the value of $MIN(i)$ is stored in \texttt{MIN[i]} and $MINDAY(i)$ in \texttt{MINDAY[i]}. We do this by scanning $p(i)$ for $i$ from $1$ to $n$, comparing each $p(i)$ with $MIN(i-1)$ and pick the less one (\texttt{MIN[1]}$ = p(1)$):
\\\tt
\\
MIN[1] = p(1)\\
MINDAY[1] = 1\\
for i from 2 to n do\\
\mbox{\hspace{2em}}MIN[i] = min\{MIN[i-1], p(i)\}\\
\mbox{\hspace{2em}}if MIN[i-1] < p(i) then\\
\mbox{\hspace{4em}}MINDAY[i] = MINDAY[i-1] else\\
\mbox{\hspace{4em}}MINDAY[i] = i\\
\mbox{\hspace{2em}}end\\
end\\
\rm

Then we calculate $OPT(i)$, $BUY(i)$ and $SELL(i)$ for $i$ from $1$ to $n$ and store the values in arrays \texttt{OPT[1..n]}, \texttt{BUY[1..n]} and \texttt{SELL[1..n]} based on Equation~1 (the base case is \texttt{OPT[1]}$ = 0$, \texttt{BUY[1]}$ = 1$ and \texttt{SELL[1]}$ = 1$):
\\\tt
\\
OPT[1] = 0\\
BUY[1] = 1\\
SELL[1] = 1\\
for i from 2 to n do\\
\mbox{\hspace{2em}}OPT[i] = max\{p(i)-MIN[i], OPT[i-1]\}\\
\mbox{\hspace{2em}}if p(i)-MIN[i] < OPT[i-1] then\\
\mbox{\hspace{4em}}BUY[i] = BUY[i-1]\\
\mbox{\hspace{4em}}SELL[i] = SELL[i-1]\\
\mbox{\hspace{2em}}else\\
\mbox{\hspace{4em}}BUY[i] = MINDAY[i]\\
\mbox{\hspace{4em}}SELL[i] = i\\
\mbox{\hspace{2em}}end\\
end\\
\rm

The final answer is \texttt{BUY[n]} and \texttt{SELL[n]}, however, if \texttt{OPT[n]}$ = 0$, which means there is no way to make money during these $n$ days, we should conclude this fact.

Since getting the value for each \texttt{MIN[i]} and \texttt{MINDAY[i]} costs constant time, the time cost of filling array \texttt{MIN[1..n]} and \texttt{MINDAY[1..n]} is $O(n)$. With \texttt{MIN[1..n]} and \texttt{MINDAY[1..n]} known, getting each \texttt{OPT[i]}, \texttt{BUY[i]} and \texttt{SELL[i]} costs constant time, so filling out \texttt{OPT[1..n]}, \texttt{BUY[1..n]} and \texttt{SELL[1..n]} also costs $O(n)$ time, so the overall time complexity of the algorithm is $O(n)$.
\end{document}
