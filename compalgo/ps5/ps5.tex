\documentclass[12pt,letterpaper]{article}
\usepackage[latin1]{inputenc}
\usepackage{amsmath}
\usepackage{amsfonts}
\usepackage{amssymb}
\usepackage{graphicx}

\newtheorem{clm}{Claim}

\author{Linyun Fu}
\title{CSCI 4020 Computer Algorithms Spring 2011\\
Problem Set 5}
\begin{document}
\maketitle
\section*{Chapter 7, Problem 7}
Consider a set of mobile computing clients in a certain town who each
need to be connected to one of several possible \emph{base stations}. We'll
suppose there are $n$ clients, with the position of each client specified
by its $(x, y)$ coordinates in the plane. There are also $k$ base stations; the
position of each of these is specified by $(x, y)$ coordinates as well.

For each client, we wish to connect it to exactly one of the base
stations. Our choice of connections is constrained in the following ways.
There is a range parameter $r$ -- a client can only be connected to a base
station that is within distance $r$. There is also a \emph{load parameter} $L$ -- no
more than $L$ clients can be connected to any single base station.

Your goal is to design a polynomial-time algorithm for the following
problem. Given the positions of a set of clients and a set of base stations,
as well as the range and load parameters, decide whether every client can
be connected simultaneously to a base station, subject to the range and
load conditions in the previous paragraph.

\section*{Answer}

\begin{figure}
%\includegraphics[width=0.7\textwidth]{7.7.eps}
\caption{Reducing the problem to network flow}
\end{figure}

\section*{Chapter 7, Problem 34}
\emph{Ad hoc networks}, made up of low-powered wireless devices, have been
proposed for situations like natural disasters in which the coordinators
of a rescue effort might want to monitor conditions in a hard-to-reach
area. The idea is that a large collection of these wireless devices could be
dropped into such an area from an airplane and then configured into a
functioning network.

Note that we're talking about (a) relatively inexpensive devices that
are (b) being dropped from an airplane into (c) dangerous territory; and
for the combination of reasons (a), (b), and (c), it becomes necessary to
include provisions for dealing with the failure of a reasonable number of
the nodes.

We'd like it to be the case that if one of the devices $v$ detects that it is in
danger of failing, it should transmit a representation of its current state to
some other device in the network. Each device has a limited transmitting
range -- say it can communicate with other devices that lie within $d$ meters
of it. Moreover, since we don't want it to try transmitting its state to a
device that has already failed, we should include some redundancy: A
device $v$ should have a set of $k$ other devices that it can potentially contact,
each within $d$ meters of it. We'll call this a \emph{back-up set} for device $v$.

\begin{itemize}
\item[(a)] Suppose you're given a set of $n$ wireless devices, with positions
represented by an $(x, y)$ coordinate pair for each. Design an algorithm
that determines whether it is possible to choose a back-up set for
each device (i.e., $k$ other devices, each within $d$ meters), with the
further property that, for some parameter $b$, no device appears in
the back-up set of more than $b$ other devices. The algorithm should
output the back-up sets themselves, provided they can be found.

\item[(b)] The idea that, for each pair of devices $v$ and $w$, there's a strict
dichotomy between being ``in range'' or ``out of range'' is a simplified
abstraction. More accurately, there's a power decay function $f(\cdot)$ that
specifies, for a pair of devices at distance $\delta$, the signal strength $f(\delta)$
that they'll be able to achieve on their wireless connection. (We'll
assume that $f(\delta)$ decreases with increasing $\delta$.)

We might want to build this into our notion of back-up sets as
follows: among the $k$ devices in the back-up set of $v$, there should
be at least one that can be reached with very high signal strength,
at least one other that can be reached with moderately high signal
strength, and so forth. More concretely, we have values $p_1 \ge p_2 \ge\cdots \ge
p_k$, so that if the back-up set for $v$ consists of devices at distances
$d_1 \le d_2 \le \cdots\le d_k$, then we should have $f(d_j) \ge p_j$ for each $j$.

Give an algorithm that determines whether it is possible to
choose a back-up set for each device subject to this more detailed
condition, still requiring that no device should appear in the back-up
set of more than $b$ other devices. Again, the algorithm should output
the back-up sets themselves, provided they can be found.
\end{itemize}

\section*{Answer}

\subsection*{Problem (a)}
\begin{figure}
%\includegraphics[width=0.7\textwidth]{7.34.eps}
\caption{Reducing the problem to network flow}
\end{figure}

\subsection*{Problem (b)}

\end{document}
