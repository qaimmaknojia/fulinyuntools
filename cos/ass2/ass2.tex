\documentclass[12pt,letterpaper]{article}
\usepackage[latin1]{inputenc}
\usepackage{amsmath}
\usepackage{amsfonts}
\usepackage{amssymb}
\usepackage{graphicx}
\usepackage{longtable}
\author{Linyun Fu\\RIN: 660929569\\Email: fula@cs.rpi.edu}
\title{Assignment 2: Paging Policies}
\begin{document}
\maketitle
\part*{Question}
Consider the following high-level code segment:

\vspace{1ex}

{\tt
\hspace{4em}for (i=0;i$<$n;i++) \{

\hspace{8em}B[i]=i*n;

\hspace{8em}A[i]=B[i]+m;

\hspace{8em}C[i]=A[i]*B[i];

\hspace{4em}\}

\hspace{4em}for (i=0;i$<$n;i++) \{

\hspace{8em}for (j=i;j$<$n;j++) \{

\hspace{12em}B[j]+=A[i]*m;

\hspace{12em}C[j]-=A[j]+B[i];

\hspace{8em}\}

\hspace{8em}B[i]+=C[n-1]+A[i];

\hspace{4em}\}
}

\begin{enumerate}
\item Using registers (denoted as $\tt R_i$), create a machine language translation of this
program loaded into the virtual address space (with the page size equal to 1024 words).
Assume that this program starts in page 0 at the internal page address 994 (each page internal
addresses start at 0 and end at 1023). The constants (like 0, 1 etc.) are placed in page 2 (at the
addresses starting at 0), the static data (for example, {\tt n}) in page 3, and each of the arrays {\tt A}, {\tt B}, {\tt C} is placed on its own page (page 4, 5 and 6 respectively).

After creating the machine code, write an expression that represents the string of page
references generated during execution of this code segment (use the parameters {\tt n} and {\tt i} and
notation (str)$^{n}$ to denote $n$ repetitions of the string str. In this string, represent n-th page by
a single decimal digit n$-$1).

\item Assume that n=11. Write a program to analyze the run-time behavior of this
code segment under the working set memory policy. In particular, your program should print a
table with rows for different values of $\Delta$ (the window size) ranging from 1 to 64, and columns
providing:
\begin{align}
\nonumber & P(\Delta), \textit{ the total number of page faults}\\
\nonumber & W(\Delta), \textit{ the average working set size, and}\\
\nonumber & F(\Delta)=\frac{P(\Delta)}{|\omega|}, \textit{ the average page fault rate.}
\end{align}
Plot the following curves: X-axis $\Delta$, Y-axis:
\begin{align}
\nonumber & P(\Delta),\\
\nonumber & W(\Delta), \textrm{ and}\\
\nonumber & \frac{1}{F(\Delta)},
\end{align}
for $\Delta$ ranging from 1 to 64.
Hint: A good solution should use the stack property of the working set policy:
\begin{itemize}
\item If a page fault occurs at $\Delta$, then it also occurs at $\Delta-1$.
\item If the working set is $W$ for $\Delta$ then it is at least $W$ for $\Delta + 1$.
\end{itemize}

\item From the last plot ($\Delta$ vs. $1/F(\Delta)$ ), explain the cause of major (at least the
largest four) knees in terms of a reference string structure (a knee of a piecewise linear curve is
point at which the slope of the curve is larger to the immediate left of this point than to the
immediate right of it).

\item In a strict implementation of the working set policy (referred to here as a strict
policy), a page that drops out of the window is removed from the memory immediately.
Compute the average and maximum number of frames assigned to the program in such a case.

\item In practice, there is a delay between marking the page as out of the working
set and the time the memory system replaces it with a new page. Assume that this delay is
short for clean pages, so they cause the page fault immediately when referenced after dropping
from the working set. For dirty pages, the delay is longer, so they will cause a page fault only
when referenced after a pages fault caused by reference to some other page. Hence, until a
page fault caused by some other page happens, any reference to a dirty page after it is out the
working set would not cause a page fault. Let's call the working set policy with such a delay a
practical working set policy. Compute the page fault rates $F(\Delta)$ with the practical working set
policy for $\Delta = 1, \cdots, 64$ and plot it together with $F(\Delta)$ for the strict implementation of the
working set policy. Explain the difference. Compute the average and maximum number of
frames assigned to the program in both cases and compare them to each other.

\item[$\bullet$] {\bf Bonus} (5 points): For each $\Delta$ and integer $n$ such that $n =$ maximum number of
frames assigned to the program under practical working set policy and under strict
implementation of the working set policy, find the fault rate under $n$ frames and OPT policy
for the string from Question 1. Plot these rates together with corresponding $F(\Delta)$ (that is maximum
$\Delta$ for which the equality $n =$ maximum number of frames assigned to the program holds) both
for the strict implementation of the working set policy and practical working set policy and
discuss the differences.
\end{enumerate}

\part*{Answer}
\section{Code Translation and Page References - Official Answer}
The machine code of the program is as follows.

\vspace{2ex}

\begin{tabular}{r|r|l|l}
Page & Internal address & Content & Page reference(s)\\
\hline
0 & 0 & \tt \#\#\# & \\
0 & $\cdots$ & $\cdots$ & \\
0 & 994 & \tt $\tt R_0\leftarrow$ ZERO & 02\\%%i
0 & 995 & \tt $\tt R_1\leftarrow$ n & 03\\%%n
0 & 996 & \tt compare $\tt R_0$,$\tt R_1$ & 0\\
0 & 997 & \tt branch\_greaterequal*+15 & 0\\%for (i=0;i<n;i++) {
0 & 998 & \tt $\tt R_2\leftarrow$ n & 03\\%%n
0 & 999 & \tt $\tt R_2\tt \leftarrow R_0*R_2$ & 0\\
0 & 1000 & \tt B[$\tt R_0$]$\tt \leftarrow R_2$ & 05\\%B[i]=i*n;
0 & 1001 & \tt $\tt R_2\leftarrow$ B[$\tt R_0$] & 05\\%%B[i]
0 & 1002 & \tt $\tt R_3\leftarrow$ m & 03\\%%m
0 & 1003 & \tt $\tt R_2\tt \leftarrow R_2+R_3$ & 0\\
0 & 1004 & \tt A[$\tt R_0$]$\tt \leftarrow R_2$ & 04\\%A[i]=B[i]+m;
0 & 1005 & \tt $\tt R_2\leftarrow$ A[$\tt R_0$] & 04\\%%A[i]
0 & 1006 & \tt $\tt R_3\leftarrow$ B[$\tt R_0$] & 05\\
0 & 1007 & \tt $\tt R_2\tt \leftarrow R_2*R_3$ & 0\\
0 & 1008 & \tt C[$\tt R_0$]$\tt \leftarrow R_2$ & 06\\%C[i]=A[i]*B[i];
0 & 1009 & \tt $\tt R_2\leftarrow$ ONE & 02\\
0 & 1010 & \tt $\tt R_0\tt \leftarrow R_0+R_2$ & 0\\%%i++
0 & 1011 & \tt branch*-15 & 0\\%}
0 & 1012 & \tt $\tt R_0\leftarrow$ ZERO & 02\\%%i
0 & 1013 & \tt $\tt R_1\leftarrow$ n & 03\\%%n
0 & 1014 & \tt compare $\tt R_0$,$\tt R_1$ & 0\\
0 & 1015 & \tt branch\_greaterequal*+32 & 0\\%for (i=0;i<n;i++) {
0 & 1016 & \tt $\tt R_2\tt \leftarrow R_0$ & 0\\%%j
0 & 1017 & \tt $\tt R_3\leftarrow$ n & 03\\%%n
0 & 1018 & \tt compare $\tt R_2$,$\tt R_3$ & 0\\
0 & 1019 & \tt branch\_greaterequal*+16 & 0\\%for (j=i;j<n;j++) {
0 & 1020 & \tt $\tt R_4\tt \leftarrow A[R_0]$ & 04\\
0 & 1021 & \tt $\tt R_5\leftarrow$ m & 03\\
0 & 1022 & \tt $\tt R_4\tt \leftarrow R_4*R_5$ & 0\\%%A[i]
0 & 1023 & \tt $\tt R_5\tt \leftarrow B[R_2]$ & 05\\%%m
1 & 0 & \tt $\tt R_4\tt \leftarrow R_5+R_4$ & 1\\
1 & 1 & $\tt B[R_2] \leftarrow R_4$ & 15\\
1 & 2 & $\tt R_4 \leftarrow A[R_2]$ & 14\\%B[j]+=A[i]*m;
1 & 3 & $\tt R_5 \leftarrow B[R_1]$ & 15\\%%C[j]
1 & 4 & $\tt R_4 \leftarrow R_4+R_5$ & 1\\%%A[j]
\end{tabular}

\begin{tabular}{r|r|l|l}
Page & Internal address & Content & Page reference(s)\\
\hline
1 & 5 & $\tt R_5 \leftarrow C[R_2]$ & 16\\%%B[i]
1 & 6 & $\tt R_4 \leftarrow R_5-R_4$ & 1\\%%A[j]+B[i]
1 & 7 & $\tt C[R_2] \leftarrow R_4$ & 16\\%C[j]-=A[j]+B[i];
1 & 8 & $\tt R_4 \leftarrow ONE$ & 12\\
1 & 9 & $\tt R_2\tt \leftarrow R_2+R_4$ & 1\\%%j++
1 & 10 & \tt branch*-16 & 1\\%}
1 & 11 & $\tt R_2\tt \leftarrow n$ & 13\\%%n
1 & 12 & $\tt R_3\tt \leftarrow ONE$ & 12\\%%1
1 & 13 & $\tt R_2\tt \leftarrow R_2-R_3$ & 1\\%%n-1
1 & 14 & $\tt R_2\tt \leftarrow C[R_2]$ & 16\\%%C[n-1]
1 & 15 & $\tt R_3\tt \leftarrow A[R_0]$ & 14\\%%A[i]
1 & 16 & $\tt R_2\tt \leftarrow R_2+R_3$ & 1\\%%C[n-1]+A[i]
1 & 17 & $\tt R_3 \leftarrow C[R_2]$ & 16\\
1 & 18 & $\tt R_2 \leftarrow R_3+R_2$ & 1\\
1 & 19 & $\tt B[R_0] \leftarrow R_2$ & 15\\%B[i]+=C[n-1]+A[i];
1 & 20 & $\tt R_2 \leftarrow ONE$ & 12\\
1 & 21 & $\tt R_0\tt \leftarrow R_0+R_2$ & 1\\%%i++
1 & 22 & \tt branch*-32 & 1\\%}
1 & $\cdots$ & $\cdots$ & \\
2 & 0 & \tt ZERO & \\
2 & 1 & \tt ONE & \\
2 & $\cdots$ & $\cdots$ & \\
3 & 0 & \tt n & \\
3 & 1 & \tt m & \\
3 & $\cdots$ & $\cdots$ & \\
4 & & \tt A & \\
5 & & \tt B & \\
6 & & \tt C & \\
\end{tabular}

\vspace{2ex}

The string of page references during execution of this code segment is as follows. 

\vspace{2ex}

\hangindent2em \,0203 \,(0003005050300404050060200)$\tt ^n$ \,00 \,110203 \,(00003 \,(00040300511514151161161211)$\tt ^n$ \,00 \,1312116141161151211) \,(00003 \,(00040300511514151161161211)$\tt ^{n-1}$ \,00 \,1312116141161151211) $\cdots$ (00003 \,(00040300511514151161161211)$^2$ \,00 \,1312116141161151211) \,(00003 \,(00040300511514151161161211) \,00 \,1312116141161151211) \,00

Actually I find a mistake here. After the first loop ending with 2 ``0''s from the instructions at 996 and 997 of page 0, the instructions at 1012 and 1013 at page 0 should be executed. So it should be ``0203'' immediately after the 2 ``0''s, not ``110203''. So the page reference string should be as follows.

\hangindent2em \,0203 \,(0003005050300404050060200)$\tt ^n$ \,00 \,0203 \,(00003 \,(00040300511514151161161211)$\tt ^n$ \,00 \,1312116141161151211) \,(00003 \,(00040300511514151161161211)$\tt ^{n-1}$ \,00 \,1312116141161151211) $\cdots$ (00003 \,(00040300511514151161161211)$^2$ \,00 \,1312116141161151211) \,(00003 \,(00040300511514151161161211) \,00 \,1312116141161151211) \,00

But if I answer the rest questions with this reference string, my answer will be different from all the other students, rendering extra burden to the TA. Therefore, in the remainder of this paper, I use the official reference string with the 2 ``1''s.

\section{Working Set Memory Policy}
By running the simulation program (see attached), I get the following results.

\begin{center}
\begin{longtable}{r|r|r|r}
\caption{$P(\Delta)$, $W(\Delta)$ and $1/F(\Delta)$ for $\Delta$ ranging from 1 to 64}\\
$\Delta$ & $P(\Delta)$ & $W(\Delta)$ & $1/F(\Delta)$\\ \hline \endfirsthead
\caption{(continued)}\\
$\Delta$ & $P(\Delta)$ & $W(\Delta)$ & $1/F(\Delta)$\\ \hline \endhead
1 & 1661 & 1.00 & 1.38\\
2 & 920 & 1.72 & 2.49\\
3 & 776 & 2.13 & 2.95\\
4 & 709 & 2.46 & 3.23\\
5 & 697 & 2.77 & 3.29\\
6 & 686 & 3.08 & 3.34\\
7 & 675 & 3.37 & 3.39\\
8 & 663 & 3.67 & 3.46\\
9 & 652 & 3.95 & 3.51\\
10 & 531 & 4.24 & 4.31\\
11 & 531 & 4.47 & 4.31\\
12 & 531 & 4.70 & 4.31\\
13 & 509 & 4.93 & 4.50\\
14 & 499 & 5.14 & 4.59\\
15 & 489 & 5.36 & 4.69\\
16 & 434 & 5.57 & 5.28\\
17 & 434 & 5.76 & 5.28\\
18 & 414 & 5.94 & 5.53\\
19 & 282 & 6.12 & 8.12\\
20 & 271 & 6.24 & 8.45\\
21 & 271 & 6.36 & 8.45\\
22 & 260 & 6.47 & 8.81\\
23 & 184 & 6.58 & 12.45\\
24 & 173 & 6.66 & 13.24\\
25 & 152 & 6.73 & 15.07\\
26 & 41 & 6.80 & 55.88\\
27 & 30 & 6.81 & 76.37\\
28 & 30 & 6.82 & 76.37\\
29 & 29 & 6.83 & 79.00\\
30 & 29 & 6.84 & 79.00\\
31 & 8 & 6.85 & 286.38\\
32 & 8 & 6.85 & 286.38\\
33 & 8 & 6.85 & 286.38\\
34 & 8 & 6.85 & 286.38\\
35 & 8 & 6.85 & 286.38\\
36 & 7 & 6.85 & 327.29\\
37 & 7 & 6.85 & 327.29\\
38 & 7 & 6.85 & 327.29\\
39 & 7 & 6.85 & 327.29\\
40 & 7 & 6.85 & 327.29\\
41 & 7 & 6.85 & 327.29\\
42 & 7 & 6.85 & 327.29\\
43 & 7 & 6.85 & 327.29\\
44 & 7 & 6.85 & 327.29\\
45 & 7 & 6.85 & 327.29\\
46 & 7 & 6.85 & 327.29\\
47 & 7 & 6.85 & 327.29\\
48 & 7 & 6.85 & 327.29\\
49 & 7 & 6.85 & 327.29\\
50 & 7 & 6.85 & 327.29\\
51 & 7 & 6.85 & 327.29\\
52 & 7 & 6.85 & 327.29\\
53 & 7 & 6.85 & 327.29\\
54 & 7 & 6.85 & 327.29\\
55 & 7 & 6.85 & 327.29\\
56 & 7 & 6.85 & 327.29\\
57 & 7 & 6.85 & 327.29\\
58 & 7 & 6.85 & 327.29\\
59 & 7 & 6.85 & 327.29\\
60 & 7 & 6.85 & 327.29\\
61 & 7 & 6.85 & 327.29\\
62 & 7 & 6.85 & 327.29\\
63 & 7 & 6.85 & 327.29\\
64 & 7 & 6.85 & 327.29\\
\end{longtable}
\end{center}

Using the data of Table 1, I plot the following curves in Figure 1, Figure 2 and Figure 3.

\begin{figure}
\begin{center}
\includegraphics[width=0.8\textwidth]{p.eps}
\caption{$\Delta$ vs. $P(\Delta)$}
\end{center}
\end{figure}

\begin{figure}
\begin{center}
\includegraphics[width=0.8\textwidth]{w.eps}
\caption{$\Delta$ vs. $W(\Delta)$}
\end{center}
\end{figure}

\begin{figure}
\begin{center}
\includegraphics[width=0.8\textwidth]{if.eps}
\caption{$\Delta$ vs. $1/F(\Delta)$}
\end{center}
\end{figure}

\section{Explaining Knees}
%explain the cause of major (at least the
%largest four) knees in terms of a reference string structure
%0203
%(0003005050300404050060200)(0003005050300404050060200)(0003005050300404050060200)(0003005050300404050060200)(0003005050300404050060200)(0003005050300404050060200)(0003005050300404050060200)(0003005050300404050060200)(0003005050300404050060200)(0003005050300404050060200)(0003005050300404050060200)
%00
%110203
%00003
%(00040300511514151161161211)(00040300511514151161161211)(00040300511514151161161211)(00040300511514151161161211)(00040300511514151161161211)(00040300511514151161161211)(00040300511514151161161211)(00040300511514151161161211)(00040300511514151161161211)(00040300511514151161161211)(00040300511514151161161211)
%001312116141161151211
%00003
%(00040300511514151161161211)(00040300511514151161161211)(00040300511514151161161211)(00040300511514151161161211)(00040300511514151161161211)(00040300511514151161161211)(00040300511514151161161211)(00040300511514151161161211)(00040300511514151161161211)(00040300511514151161161211)
%001312116141161151211
%00003
%(00040300511514151161161211)(00040300511514151161161211)(00040300511514151161161211)(00040300511514151161161211)(00040300511514151161161211)(00040300511514151161161211)(00040300511514151161161211)(00040300511514151161161211)(00040300511514151161161211)
%001312116141161151211
%00003
%(00040300511514151161161211)(00040300511514151161161211)(00040300511514151161161211)(00040300511514151161161211)(00040300511514151161161211)(00040300511514151161161211)(00040300511514151161161211)(00040300511514151161161211)
%001312116141161151211

%00
%110203[00003(00040300511514151161161211)n-i00
%             1312116141161151211]n00

From Table 1 and Figure 3, we can see that the major knees are at $\Delta=26, \Delta=27, \Delta=31$ and $\Delta=36$. The cause of these knees are as follows.
\begin{itemize}
\item When $\Delta$ is increased from 25 to 26, the following page references will not cause page faults, yet they do cause page faults before $\Delta$ increases. 
\begin{itemize}
\item the first reference of Page 4 in the inner loop;
\item from the second round of the inner loop, all the references of Page 3 in the inner loop;
\item from the second round of the inner loop, all the references of Page 2 in the inner loop.
\end{itemize}
\item When $\Delta$ is increased from 26 to 27, the following page references will not cause page faults, yet they do cause page faults before $\Delta$ increases.
\begin{itemize}
\item the last reference of Page 5 in the second loop.
\end{itemize}
\item When $\Delta$ is increased from 30 to 31, the following page references will not cause page faults, yet they do cause page faults before $\Delta$ increases.
\begin{itemize}
\item in the first round of the inner loop, the reference of Page 2;
\item from the second round of the inner loop, all the references of the earlier Page 6 in the inner loop.
\end{itemize}
\item When $\Delta$ is increased from 35 to 36, the following page references will not cause page faults, yet they do cause page faults before $\Delta$ increases.
\begin{itemize}
\item in the first round of the inner loop, the reference of the earlier Page 6.
\end{itemize}
\end{itemize}

\section{Strict Policy}
By running the simulation program (see attached), I get the following results. Since the number of assigned frames is the same as the working set size under the strict policy, I use $W_{avg}$ and $W_{max}$ to denote average and maximum number of frames assigned to the program.

\begin{center}
\begin{longtable}{r|r|r}
\caption{Average and maximum number of assigned frames for $\Delta$ ranging from 1 to 64}\\
$\Delta$ & $W_{avg}$ & $W_{max}$\\ \hline \endfirsthead
\caption{(continued)}\\
$\Delta$ & $W_{avg}$ & $W_{max}$\\ \hline \endhead
1 & 1.00 & 1.00\\
2 & 1.72 & 2.00\\
3 & 2.13 & 3.00\\
4 & 2.46 & 3.00\\
5 & 2.77 & 4.00\\
6 & 3.08 & 4.00\\
7 & 3.37 & 5.00\\
8 & 3.67 & 5.00\\
9 & 3.95 & 5.00\\
10 & 4.24 & 6.00\\
11 & 4.47 & 6.00\\
12 & 4.70 & 6.00\\
13 & 4.93 & 6.00\\
14 & 5.14 & 7.00\\
15 & 5.36 & 7.00\\
16 & 5.57 & 7.00\\
17 & 5.76 & 7.00\\
18 & 5.94 & 7.00\\
19 & 6.12 & 7.00\\
20 & 6.24 & 7.00\\
21 & 6.36 & 7.00\\
22 & 6.47 & 7.00\\
23 & 6.58 & 7.00\\
24 & 6.66 & 7.00\\
25 & 6.73 & 7.00\\
26 & 6.80 & 7.00\\
27 & 6.81 & 7.00\\
28 & 6.82 & 7.00\\
29 & 6.83 & 7.00\\
30 & 6.84 & 7.00\\
31 & 6.85 & 7.00\\
32 & 6.85 & 7.00\\
33 & 6.85 & 7.00\\
34 & 6.85 & 7.00\\
35 & 6.85 & 7.00\\
36 & 6.85 & 7.00\\
37 & 6.85 & 7.00\\
38 & 6.85 & 7.00\\
39 & 6.85 & 7.00\\
40 & 6.85 & 7.00\\
41 & 6.85 & 7.00\\
42 & 6.85 & 7.00\\
43 & 6.85 & 7.00\\
44 & 6.85 & 7.00\\
45 & 6.85 & 7.00\\
46 & 6.85 & 7.00\\
47 & 6.85 & 7.00\\
48 & 6.85 & 7.00\\
49 & 6.85 & 7.00\\
50 & 6.85 & 7.00\\
51 & 6.85 & 7.00\\
52 & 6.85 & 7.00\\
53 & 6.85 & 7.00\\
54 & 6.85 & 7.00\\
55 & 6.85 & 7.00\\
56 & 6.85 & 7.00\\
57 & 6.85 & 7.00\\
58 & 6.85 & 7.00\\
59 & 6.85 & 7.00\\
60 & 6.85 & 7.00\\
61 & 6.85 & 7.00\\
62 & 6.85 & 7.00\\
63 & 6.85 & 7.00\\
64 & 6.85 & 7.00\\
\end{longtable}
\end{center}

\section{Practical Working Set Policy}
By running the simulation program (see attached), I get the following results. I use $F_s(\Delta)$ and $F_p(\Delta)$ to denote the page fault rates with the strict policy and the practical policy, respectively; I use $W_{smax}(\Delta)$ and $W_{pmax}(\Delta)$ to denote the maximum number of frames assigned with the strict policy and the practical policy, respectively; I use $W_{savg}(\Delta)$ and $W_{pavg}(\Delta)$ to denote the average number of frames assigned with the strict policy and the practical policy, respectively.

\begin{center}
\begin{longtable}{r|r|r|r|r|r|r}
\caption{Page fault rate and maximum/average assigned frames for $\Delta$ ranging from 1 to 64 with strict and practical policies}\\
$\Delta$ & $F_s(\Delta)$ & $F_p(\Delta)$ & $W_{smax}(\Delta)$ & $W_{pmax}(\Delta)$ & $W_{savg}(\Delta)$ & $W_{pavg}(\Delta)$\\ \hline \endfirsthead
\caption{(continued)}\\
$\Delta$ & $F_s(\Delta)$ & $F_p(\Delta)$ & $W_{smax}(\Delta)$ & $W_{pmax}(\Delta)$ & $W_{savg}(\Delta)$ & $W_{pavg}(\Delta)$\\ \hline \endhead
1 & 0.73 & 0.72 & 1.00 & 4.00 & 1.00 & 3.73\\
2 & 0.40 & 0.37 & 2.00 & 5.00 & 1.72 & 4.20\\
3 & 0.34 & 0.31 & 3.00 & 6.00 & 2.13 & 4.36\\
4 & 0.31 & 0.31 & 3.00 & 6.00 & 2.46 & 4.52\\
5 & 0.30 & 0.30 & 4.00 & 6.00 & 2.77 & 3.99\\
6 & 0.30 & 0.30 & 4.00 & 6.00 & 3.08 & 4.18\\
7 & 0.29 & 0.29 & 5.00 & 6.00 & 3.37 & 4.37\\
8 & 0.29 & 0.28 & 5.00 & 7.00 & 3.67 & 4.55\\
9 & 0.28 & 0.28 & 5.00 & 7.00 & 3.95 & 4.73\\
10 & 0.23 & 0.23 & 6.00 & 7.00 & 4.24 & 4.91\\
11 & 0.23 & 0.23 & 6.00 & 7.00 & 4.47 & 5.06\\
12 & 0.23 & 0.23 & 6.00 & 7.00 & 4.70 & 4.98\\
13 & 0.22 & 0.22 & 6.00 & 7.00 & 4.93 & 5.17\\
14 & 0.22 & 0.22 & 7.00 & 7.00 & 5.14 & 5.35\\
15 & 0.21 & 0.21 & 7.00 & 7.00 & 5.36 & 5.53\\
16 & 0.19 & 0.19 & 7.00 & 7.00 & 5.57 & 5.71\\
17 & 0.19 & 0.19 & 7.00 & 7.00 & 5.76 & 5.86\\
18 & 0.18 & 0.18 & 7.00 & 7.00 & 5.94 & 6.02\\
19 & 0.12 & 0.12 & 7.00 & 7.00 & 6.12 & 6.12\\
20 & 0.12 & 0.12 & 7.00 & 7.00 & 6.24 & 6.24\\
21 & 0.12 & 0.12 & 7.00 & 7.00 & 6.36 & 6.36\\
22 & 0.11 & 0.11 & 7.00 & 7.00 & 6.47 & 6.47\\
23 & 0.08 & 0.08 & 7.00 & 7.00 & 6.58 & 6.58\\
24 & 0.08 & 0.08 & 7.00 & 7.00 & 6.66 & 6.66\\
25 & 0.07 & 0.07 & 7.00 & 7.00 & 6.73 & 6.73\\
26 & 0.02 & 0.02 & 7.00 & 7.00 & 6.80 & 6.80\\
27 & 0.01 & 0.01 & 7.00 & 7.00 & 6.81 & 6.81\\
28 & 0.01 & 0.01 & 7.00 & 7.00 & 6.82 & 6.82\\
29 & 0.01 & 0.01 & 7.00 & 7.00 & 6.83 & 6.83\\
30 & 0.01 & 0.01 & 7.00 & 7.00 & 6.84 & 6.84\\
31 & 0.00 & 0.00 & 7.00 & 7.00 & 6.85 & 6.85\\
32 & 0.00 & 0.00 & 7.00 & 7.00 & 6.85 & 6.85\\
33 & 0.00 & 0.00 & 7.00 & 7.00 & 6.85 & 6.85\\
34 & 0.00 & 0.00 & 7.00 & 7.00 & 6.85 & 6.85\\
35 & 0.00 & 0.00 & 7.00 & 7.00 & 6.85 & 6.85\\
36 & 0.00 & 0.00 & 7.00 & 7.00 & 6.85 & 6.85\\
37 & 0.00 & 0.00 & 7.00 & 7.00 & 6.85 & 6.85\\
38 & 0.00 & 0.00 & 7.00 & 7.00 & 6.85 & 6.85\\
39 & 0.00 & 0.00 & 7.00 & 7.00 & 6.85 & 6.85\\
40 & 0.00 & 0.00 & 7.00 & 7.00 & 6.85 & 6.85\\
41 & 0.00 & 0.00 & 7.00 & 7.00 & 6.85 & 6.85\\
42 & 0.00 & 0.00 & 7.00 & 7.00 & 6.85 & 6.85\\
43 & 0.00 & 0.00 & 7.00 & 7.00 & 6.85 & 6.85\\
44 & 0.00 & 0.00 & 7.00 & 7.00 & 6.85 & 6.85\\
45 & 0.00 & 0.00 & 7.00 & 7.00 & 6.85 & 6.85\\
46 & 0.00 & 0.00 & 7.00 & 7.00 & 6.85 & 6.85\\
47 & 0.00 & 0.00 & 7.00 & 7.00 & 6.85 & 6.85\\
48 & 0.00 & 0.00 & 7.00 & 7.00 & 6.85 & 6.85\\
49 & 0.00 & 0.00 & 7.00 & 7.00 & 6.85 & 6.85\\
50 & 0.00 & 0.00 & 7.00 & 7.00 & 6.85 & 6.85\\
51 & 0.00 & 0.00 & 7.00 & 7.00 & 6.85 & 6.85\\
52 & 0.00 & 0.00 & 7.00 & 7.00 & 6.85 & 6.85\\
53 & 0.00 & 0.00 & 7.00 & 7.00 & 6.85 & 6.85\\
54 & 0.00 & 0.00 & 7.00 & 7.00 & 6.85 & 6.85\\
55 & 0.00 & 0.00 & 7.00 & 7.00 & 6.85 & 6.85\\
56 & 0.00 & 0.00 & 7.00 & 7.00 & 6.85 & 6.85\\
57 & 0.00 & 0.00 & 7.00 & 7.00 & 6.85 & 6.85\\
58 & 0.00 & 0.00 & 7.00 & 7.00 & 6.85 & 6.85\\
59 & 0.00 & 0.00 & 7.00 & 7.00 & 6.85 & 6.85\\
60 & 0.00 & 0.00 & 7.00 & 7.00 & 6.85 & 6.85\\
61 & 0.00 & 0.00 & 7.00 & 7.00 & 6.85 & 6.85\\
62 & 0.00 & 0.00 & 7.00 & 7.00 & 6.85 & 6.85\\
63 & 0.00 & 0.00 & 7.00 & 7.00 & 6.85 & 6.85\\
64 & 0.00 & 0.00 & 7.00 & 7.00 & 6.85 & 6.85\\
\end{longtable}
\end{center}

According to the first three columns of Table 3, I plot the curves in Figure 4.

\begin{figure}
\begin{center}
\includegraphics[width=0.8\textwidth]{fsfp.eps}
\caption{$\Delta$ vs. $F_s(\Delta)$ and $F_p(\Delta)$}
\end{center}
\end{figure}

From Figure 4, we can see that the page fault rate for the strict policy is always slightly greater than or equal to the practical policy. This is because of the delay of dirty page replacements.

By comparing the last 4 columns of Table 3, we can see that the average and maximum frames assigned for the strict policy is are always less than or equal to the practical policy. This is also because of the delay of dirty page replacements.

\section*{Bonus}
By running the simulation program (see attached), I get the following results.

\begin{center}
\begin{longtable}{r|r|r|r|r|r|r}
\caption{Page fault rate and maximum/average assigned frames for $\Delta$ ranging from 1 to 64 with strict and practical policies}\\
$\Delta$ & $n$ & $F_s(\Delta)$ & $F_p(\Delta)$ & $F_{OPT}(\Delta)$\\ \hline \endfirsthead
\caption{(continued)}\\
$\Delta$ & $n$ & $F_s(\Delta)$ & $F_p(\Delta)$ & $F_{OPT}(\Delta)$\\ \hline \endhead
1 & 4 & 0.73 & 0.72 & 0.15\\
2 & 5 & 0.40 & 0.37 & 0.10\\
3 & 6 & 0.34 & 0.31 & 0.05\\
4 & 6 & 0.31 & 0.31 & 0.05\\
5 & 6 & 0.30 & 0.30 & 0.05\\
6 & 6 & 0.30 & 0.30 & 0.05\\
7 & 6 & 0.29 & 0.29 & 0.05\\
8 & 7 & 0.29 & 0.28 & 0.00\\
9 & 7 & 0.28 & 0.28 & 0.00\\
10 & 7 & 0.23 & 0.23 & 0.00\\
11 & 7 & 0.23 & 0.23 & 0.00\\
12 & 7 & 0.23 & 0.23 & 0.00\\
13 & 7 & 0.22 & 0.22 & 0.00\\
14 & 7 & 0.22 & 0.22 & 0.00\\
15 & 7 & 0.21 & 0.21 & 0.00\\
16 & 7 & 0.19 & 0.19 & 0.00\\
17 & 7 & 0.19 & 0.19 & 0.00\\
18 & 7 & 0.18 & 0.18 & 0.00\\
19 & 7 & 0.12 & 0.12 & 0.00\\
20 & 7 & 0.12 & 0.12 & 0.00\\
21 & 7 & 0.12 & 0.12 & 0.00\\
22 & 7 & 0.11 & 0.11 & 0.00\\
23 & 7 & 0.08 & 0.08 & 0.00\\
24 & 7 & 0.08 & 0.08 & 0.00\\
25 & 7 & 0.07 & 0.07 & 0.00\\
26 & 7 & 0.02 & 0.02 & 0.00\\
27 & 7 & 0.01 & 0.01 & 0.00\\
28 & 7 & 0.01 & 0.01 & 0.00\\
29 & 7 & 0.01 & 0.01 & 0.00\\
30 & 7 & 0.01 & 0.01 & 0.00\\
31 & 7 & 0.00 & 0.00 & 0.00\\
32 & 7 & 0.00 & 0.00 & 0.00\\
33 & 7 & 0.00 & 0.00 & 0.00\\
34 & 7 & 0.00 & 0.00 & 0.00\\
35 & 7 & 0.00 & 0.00 & 0.00\\
36 & 7 & 0.00 & 0.00 & 0.00\\
37 & 7 & 0.00 & 0.00 & 0.00\\
38 & 7 & 0.00 & 0.00 & 0.00\\
39 & 7 & 0.00 & 0.00 & 0.00\\
40 & 7 & 0.00 & 0.00 & 0.00\\
41 & 7 & 0.00 & 0.00 & 0.00\\
42 & 7 & 0.00 & 0.00 & 0.00\\
43 & 7 & 0.00 & 0.00 & 0.00\\
44 & 7 & 0.00 & 0.00 & 0.00\\
45 & 7 & 0.00 & 0.00 & 0.00\\
46 & 7 & 0.00 & 0.00 & 0.00\\
47 & 7 & 0.00 & 0.00 & 0.00\\
48 & 7 & 0.00 & 0.00 & 0.00\\
49 & 7 & 0.00 & 0.00 & 0.00\\
50 & 7 & 0.00 & 0.00 & 0.00\\
51 & 7 & 0.00 & 0.00 & 0.00\\
52 & 7 & 0.00 & 0.00 & 0.00\\
53 & 7 & 0.00 & 0.00 & 0.00\\
54 & 7 & 0.00 & 0.00 & 0.00\\
55 & 7 & 0.00 & 0.00 & 0.00\\
56 & 7 & 0.00 & 0.00 & 0.00\\
57 & 7 & 0.00 & 0.00 & 0.00\\
58 & 7 & 0.00 & 0.00 & 0.00\\
59 & 7 & 0.00 & 0.00 & 0.00\\
60 & 7 & 0.00 & 0.00 & 0.00\\
61 & 7 & 0.00 & 0.00 & 0.00\\
62 & 7 & 0.00 & 0.00 & 0.00\\
63 & 7 & 0.00 & 0.00 & 0.00\\
64 & 7 & 0.00 & 0.00 & 0.00\\
\end{longtable}
\end{center}

From the data in Table 4, I plot the curves in Figure 5.

\begin{figure}
\begin{center}
\includegraphics[width=0.8\textwidth]{fopt.eps};
\end{center}
\end{figure}

From Table 4 and Figure 5, we can see that the practical policy always has a slightly lower or equal page fault rate compared with the strict policy, and the OPT policy has a much lower page fault rate than the other two.
\end{document}

