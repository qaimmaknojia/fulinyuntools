\documentclass[12pt,letterpaper]{article}
\usepackage[latin1]{inputenc}
\usepackage{amsmath}
\usepackage{amsfonts}
\usepackage{amssymb}
\usepackage{graphicx}
\author{Linyun Fu}
\title{CSCI.4430/6969 Programming Languages Spring 2011\\
Programming Assignment \#1 -- Solving Lights Out in Prolog}
\begin{document}
\maketitle
\part*{Question}
Your goal is to solve the game Lights Out in Prolog. The game consists of a 5 by 5 grid of BLACK or WHITE points. When the game starts, a set of points will be set to BLACK, the others will show WHITE. Turning one point will toggle it and the four surrounding neighbor points (up, down, left, right; no wrapping to other sides of the board) WHITE and BLACK (WHITE $\rightarrow$ BLACK, BLACK $\rightarrow$ WHITE). The final goal is to make all the points in the grid BLACK. 

\section*{Input and Output Format}
The program should output a series of moves needed to ``win'' the game, i.e. to meet the condition that all lights are out (BLACK).

For example an initial board may be:\\
\tt BBBBB\\
BBBBB\\
BBBBB\\
BBBBB\\
WBBBW

\rm Where ``B'' means it is BLACK, ``W'' means it is WHITE. Your program should output the following moves and the ending board:\\
\tt 1,1\\
1,2\\
2,3\\
3,1\\
3,3\\
3,4\\
4,1\\
4,3\\
4,5\\
5,4\\
5,5\\
\\
BBBBB\\
BBBBB\\
BBBBB\\
BBBBB\\
BBBBB
\rm
\section*{Extra Credit (10\% bonus).}

Extend your program to solve the puzzle using the fewest possible turns.

\part*{Answer}
If the problem has a solution that contains pressing the same cell for two or more times, we can always reduce the number of moves by eliminating double moves on the same cell. So we can reduce the search space by restricting each cell to be pressed at most once.

We use $C_{ij}$ to denote the initial status of the cell at the $i$-th row and the $j$-th column. If it's white, $C_{ij}=1$; if it's black, $C_{ij}=0$. We use $N_{ij}$ to denote the number of moves applied on the cell at the $i$-th row and the $j$-th column, so each $N_{ij}$ is either 0 or 1.

Then the problem is reduced to solving the following equation system for $N_{ij}$, with $1\le i \le 5$ and $1\le j \le 5$.
\begin{align}
N_{11} \oplus N_{12} \oplus N_{21} & = C_{11}\\
N_{11} \oplus N_{12} \oplus N_{13} \oplus N_{22} & = C_{12}\\
N_{12} \oplus N_{13} \oplus N_{14} \oplus N_{23} & = C_{13}\\
N_{13} \oplus N_{14} \oplus N_{15} \oplus N_{24} & = C_{14}\\
N_{14} \oplus N_{15} \oplus N_{25} & = C_{15}\\
N_{11} \oplus N_{21} \oplus N_{22} \oplus N_{31} & = C_{21}\\ 
N_{12} \oplus N_{21} \oplus N_{22} \oplus N_{23} \oplus N_{32} & = C_{22}\\
N_{13} \oplus N_{22} \oplus N_{23} \oplus N_{24} \oplus N_{33} & = C_{23}\\ 
N_{14} \oplus N_{23} \oplus N_{24} \oplus N_{25} \oplus N_{34} & = C_{24}\\ 
N_{15} \oplus N_{24} \oplus N_{25} \oplus N_{35} & = C_{25}\\ 
N_{21} \oplus N_{31} \oplus N_{32} \oplus N_{41} & = C_{31}\\ 
N_{22} \oplus N_{31} \oplus N_{32} \oplus N_{33} \oplus N_{42} & = C_{32}\\ 
N_{23} \oplus N_{32} \oplus N_{33} \oplus N_{34} \oplus N_{43} & = C_{33}\\ 
N_{24} \oplus N_{33} \oplus N_{34} \oplus N_{35} \oplus N_{44} & = C_{34}\\ 
N_{25} \oplus N_{34} \oplus N_{35} \oplus N_{45} & = C_{35}\\ 
N_{31} \oplus N_{41} \oplus N_{42} \oplus N_{51} & = C_{41}\\ 
N_{32} \oplus N_{41} \oplus N_{42} \oplus N_{43} \oplus N_{52} & = C_{42}\\ 
N_{33} \oplus N_{42} \oplus N_{43} \oplus N_{44} \oplus N_{53} & = C_{43}\\ 
N_{34} \oplus N_{43} \oplus N_{44} \oplus N_{45} \oplus N_{54} & = C_{44}\\ 
N_{35} \oplus N_{44} \oplus N_{45} \oplus N_{55} & = C_{45}\\ 
N_{41} \oplus N_{51} \oplus N_{52} & = C_{51}\\ 
N_{42} \oplus N_{51} \oplus N_{52} \oplus N_{53} & = C_{52}\\ 
N_{43} \oplus N_{52} \oplus N_{53} \oplus N_{54} & = C_{53}\\ 
N_{44} \oplus N_{53} \oplus N_{54} \oplus N_{55} & = C_{54}\\ 
N_{45} \oplus N_{54} \oplus N_{55} & = C_{55}
\end{align}
where $\oplus$ is defined as
\begin{align}
0\oplus 0 = 0\\
0\oplus 1 = 1\\
1\oplus 0 = 1\\
1\oplus 1 = 0
\end{align}
Equation~1 through 25 yield
\begin{align}
0 = & C_{11} \oplus C_{13} \oplus C_{15} \oplus C_{21} \oplus C_{23} \oplus C_{25} \\\nonumber
& \oplus C_{41} \oplus C_{43} \oplus C_{45} \oplus C_{51} \oplus C_{53} \oplus C_{55}\\
0 = & C_{12} \oplus C_{13} \oplus C_{14} \oplus C_{21} \oplus C_{23} \oplus C_{25} \\\nonumber
& \oplus C_{31} \oplus C_{32} \oplus C_{34} \oplus C_{35} \oplus C_{41} \oplus C_{43} \\\nonumber
& \oplus C_{45} \oplus C_{52} \oplus C_{53} \oplus C_{54}\\
N_{11} = & C_{21} \oplus C_{31} \oplus C_{32} \oplus C_{41} \oplus C_{43} \oplus C_{52} \\\nonumber
& \oplus C_{53} \oplus C_{54} \\\nonumber
& \oplus N_{55}\\
N_{12} = & C_{22} \oplus C_{31} \oplus C_{32} \oplus C_{33} \oplus C_{44} \oplus C_{51} \\\nonumber
& \oplus C_{52} \oplus C_{54} \oplus C_{55} \\\nonumber
& \oplus N_{54}\\
N_{13} = & C_{14} \oplus C_{24} \oplus C_{25} \oplus C_{33} \oplus C_{34} \oplus C_{42} \\\nonumber
& \oplus C_{44} \oplus C_{51} \oplus C_{52} \oplus C_{53} \oplus C_{55} \\\nonumber
& \oplus N_{54} \oplus N_{55}\\
N_{14} = & C_{13} \oplus C_{14} \oplus C_{15} \oplus C_{22} \oplus C_{31} \oplus C_{32} \\\nonumber
& \oplus C_{34} \oplus C_{35} \oplus C_{42} \oplus C_{53} \oplus C_{54} \oplus C_{55} \\\nonumber
& \oplus N_{54}\\
N_{15} = & C_{14} \oplus C_{15} \oplus C_{23} \oplus C_{25} \oplus C_{32} \oplus C_{33} \\\nonumber
& \oplus C_{34} \oplus C_{41} \oplus C_{51} \oplus C_{53} \oplus C_{54} \\\nonumber
& \oplus N_{55}\\
N_{21} = & C_{13} \oplus C_{15} \oplus C_{22} \oplus C_{23} \oplus C_{25} \oplus C_{33} \\\nonumber
& \oplus C_{44} \oplus C_{45} \\\nonumber
& \oplus N_{54} \oplus N_{55}\\
N_{22} = & C_{13} \oplus C_{22} \oplus C_{23} \oplus C_{24} \oplus C_{31} \oplus C_{32} \\\nonumber
& \oplus C_{35} \oplus C_{42} \oplus C_{45} \oplus C_{53} \oplus C_{54}\\
N_{23} = & C_{15} \oplus C_{24} \oplus C_{25} \oplus C_{35} \oplus C_{55} \\\nonumber
& \oplus N_{54} \oplus N_{55}
\end{align}
\begin{align}
N_{24} = & C_{13} \oplus C_{22} \oplus C_{23} \oplus C_{24} \oplus C_{31} \oplus C_{34} \\\nonumber
& \oplus C_{35} \oplus C_{41} \oplus C_{44} \oplus C_{52} \oplus C_{53}\\
N_{25} = & C_{13} \oplus C_{15} \oplus C_{22} \oplus C_{23} \oplus C_{25} \oplus C_{31} \\\nonumber
& \oplus C_{33} \oplus C_{35} \oplus C_{41} \oplus C_{42} \oplus C_{51} \oplus C_{55} \\\nonumber
& \oplus N_{54} \oplus N_{55}\\
N_{31} = & C_{15} \oplus C_{24} \oplus C_{25} \oplus C_{33} \oplus C_{35} \oplus C_{41} \\\nonumber
& \oplus C_{42} \oplus C_{43} \oplus C_{44} \oplus C_{52} \\\nonumber
& \oplus N_{54}\\
N_{32} = & C_{42} \oplus C_{51} \oplus C_{52} \oplus C_{53} \\\nonumber
& \oplus N_{54}\\
N_{33} = & C_{14} \oplus C_{15} \oplus C_{23} \oplus C_{32} \oplus C_{33} \oplus C_{35} \\\nonumber
& \oplus C_{41} \oplus C_{45} \oplus C_{51} \oplus C_{53} \oplus C_{54}\\
N_{34} = & C_{13} \oplus C_{14} \oplus C_{15} \oplus C_{22} \oplus C_{24} \oplus C_{31} \\\nonumber
& \oplus C_{32} \oplus C_{33} \oplus C_{44} \oplus C_{51} \oplus C_{52} \oplus C_{54} \oplus C_{55} \\\nonumber
& \oplus N_{54}\\
N_{35} = & C_{14} \oplus C_{23} \oplus C_{24} \oplus C_{25} \oplus C_{32} \oplus C_{41} \\\nonumber
& \oplus C_{42} \oplus C_{44} \oplus C_{52} \oplus C_{54} \oplus C_{55} \\\nonumber
& \oplus N_{54}\\
N_{41} = & C_{13} \oplus C_{22} \oplus C_{23} \oplus C_{24} \oplus C_{31} \oplus C_{35} \\\nonumber
& \oplus C_{41} \oplus C_{43} \oplus C_{45} \oplus C_{51} \oplus C_{53} \\\nonumber
& \oplus N_{54} \oplus N_{55}\\
N_{42} = & C_{13} \oplus C_{14} \oplus C_{22} \oplus C_{25} \oplus C_{31} \oplus C_{32} \\\nonumber
& \oplus C_{35} \oplus C_{42} \oplus C_{43} \oplus C_{44} \oplus C_{53}\\
N_{43} = & C_{13} \oplus C_{15} \oplus C_{22} \oplus C_{23} \oplus C_{25} \oplus C_{31} \\\nonumber
& \oplus C_{33} \oplus C_{41} \oplus C_{42} \oplus C_{44} \oplus C_{45} \oplus C_{51} \\\nonumber
& \oplus N_{54} \oplus N_{55}\\
N_{44} = & C_{14} \oplus C_{23} \oplus C_{24} \oplus C_{25} \oplus C_{32} \oplus C_{41} \\\nonumber
& \oplus C_{42} \oplus C_{44} \oplus C_{45} \oplus C_{52} \oplus C_{54}\\
N_{45} = & C_{55} \\\nonumber
& \oplus N_{54} \oplus N_{55}
\end{align}
\begin{align}
N_{51} = & C_{14} \oplus C_{15} \oplus C_{23} \oplus C_{32} \oplus C_{33} \oplus C_{35} \\\nonumber
& \oplus C_{41} \oplus C_{43} \oplus C_{45} \oplus C_{51} \oplus C_{52} \\\nonumber
& \oplus N_{55}\\
N_{52} = & C_{13} \oplus C_{14} \oplus C_{15} \oplus C_{22} \oplus C_{24} \oplus C_{31} \\\nonumber
& \oplus C_{32} \oplus C_{33} \oplus C_{51} \oplus C_{52} \oplus C_{53}\\\nonumber
& \oplus N_{54}\\
N_{53} = & C_{14} \oplus C_{23} \oplus C_{24} \oplus C_{25} \oplus C_{32} \oplus C_{41} \\\nonumber
& \oplus C_{42} \oplus C_{44} \oplus C_{45} \oplus C_{52}\\\nonumber
& \oplus N_{54} \oplus N_{55}
\end{align}
So a solution exists if $C_{11}$ through $C_{55}$ satisfy Equation~30 and 31. When this is the case, the equation system has 4 different solutions because there are only 23 equations but there are 25 variables ($N_{11}$ through $N_{55}$) and the coefficient vectors of these 23 equations are linear independent, so two of the variables (e.g., $N_{54}$ and $N_{55}$) are free to be either 0 or 1, yielding 4 different solutions.

Based on the above idea, if the board is solvable, the solution with the fewest moves must be among the four possible solutions. My Prolog program (see the attached file named ``lightsout.pl'') searches through all possible values of all the variables to find solutions, so if you type semicolon for enough times, the program will finally find all the four solutions. For the sample input
\tt\\
BBBBB\\
BBBBB\\
BBBBB\\
BBBBB\\
WBBBW
\rm\\
these solutions are
\tt\\
1,4\\
1,5\\
2,3\\
3,2\\
3,3\\
3,5\\
4,1\\
4,3\\
4,5\\
5,1\\
5,2\\
true ;\\
\\
1,2\\
1,3\\
1,5\\
2,1\\
2,5\\
3,1\\
3,3\\
3,4\\
5,1\\
5,3\\
5,4\\
true ;\\
\\
1,1\\
1,3\\
1,4\\
2,1\\
2,5\\
3,2\\
3,3\\
3,5\\
5,2\\
5,3\\
5,5\\
true ;\\
\\
1,1\\
1,2\\
2,3\\
3,1\\
3,3\\
3,4\\
4,1\\
4,3\\
4,5\\
5,4\\
5,5\\
true .\\
\rm\\
\\
Each of the solutions contains 11 moves. They are all optimal solutions.
\end{document}

